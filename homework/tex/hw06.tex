\documentclass[12pt,reqno]{amsart}
\usepackage[top=2cm, left=2cm,right=2cm,bottom=2cm]{geometry}
\renewcommand{\baselinestretch}{1.2}
\usepackage{amsmath}
\usepackage{amssymb}
\usepackage{color,hyperref,enumerate,multicol}
\definecolor{darkblue}{rgb}{0.0,0.0,0.3}
\hypersetup{colorlinks,breaklinks,
            linkcolor=darkblue,urlcolor=darkblue,
            anchorcolor=darkblue,citecolor=darkblue}
            
\usepackage{algorithm}
\usepackage{algorithmic}
\pagestyle{empty}
\newcommand{\N}{\ensuremath{\mathbb{N}}}
\newcommand{\Z}{\ensuremath{\mathbb{Z}}}
\newcommand{\R}{\ensuremath{\mathbb{R}}}
\newcommand{\meet}{\ensuremath{\wedge}}
\newcommand{\Meet}{\ensuremath{\bigwedge}}
\newcommand{\join}{\ensuremath{\vee}}
\renewcommand{\emptyset}{\ensuremath{\varnothing}}
\renewcommand{\subset}{\ensuremath{\subsetneq}}
\newcommand{\boldemph}{\emph}
\newcommand{\lcm}{\operatorname{lcm}}

\newcommand{\probskip}{\vskip1cm}

\begin{document}
\thispagestyle{empty}

\noindent \textbf{Math 301} \hskip5cm {\bf Homework 6} \hfill {\bf Fall 2014}
\vskip1cm
\noindent {\bf Chapter 5:}   1bd, 3bd, 4, 6, 17, 18, 27.\\  
   Additional suggested exercises: 29, 31, 32, 33.  \\
{\bf Due date:} Friday, 10/10

\medskip

\noindent (Exercise numbers correspond to the printed textbook, generated from 2013/08/16 source files.)

\medskip

\begin{enumerate}[{\bf 1.}]

%% 1 %%%%%%%%%%%%%%%%%%%%%%%%%%%%%%%%%%%%%%%%%%%%%%%%
\item %1
Write the following permutations in cycle notation.
\begin{multicols}{2}
\begin{enumerate}
 
\item
\[
\begin{pmatrix}
1 & 2 & 3 & 4 & 5 \\
2 & 4 & 1 & 5 & 3
\end{pmatrix}
\]

\item
\[
\begin{pmatrix}
1 & 2 & 3 & 4 & 5 \\
4 & 2 & 5 & 1 & 3
\end{pmatrix}
\]

\item
\[
\begin{pmatrix}
1 & 2 & 3 & 4 & 5 \\
3 & 5 & 1 & 4 & 2
\end{pmatrix}
\]

\item
\[
\begin{pmatrix}
1 & 2 & 3 & 4 & 5 \\
1 & 4 & 3 & 2 & 5
\end{pmatrix}
\]

\end{enumerate}
\end{multicols}

\probskip
 
%% 3 %%%%%%%%%%%%%%%%%%%%%%%%%%%%%%%%%%%%%%%%%%%%%%%%
\item[{\bf 3.}] 
Express the following permutations as products of transpositions and
identify them as even or odd. 
\begin{multicols}{2}
\begin{enumerate}
 
\item
$(14356)$

 \item
$(156)(234)$
 
 \item
$(1426)(142)$
 
 \item
$(17254)(1423)(154632)$
 
 \item
$(142637)$
 
\end{enumerate}
\end{multicols}

\probskip
 
%% 4 %%%%%%%%%%%%%%%%%%%%%%%%%%%%%%%%%%%%%%%%%%%%%%%%
\item[{\bf 4.}] 
Find $(a_1, a_2, \ldots, a_n)^{-1}$.

\probskip
 
%% 6 %%%%%%%%%%%%%%%%%%%%%%%%%%%%%%%%%%%%%%%%%%%%%%%%
\item[{\bf 6.}] 
Find all of the subgroups in $A_4$. What is the order of each
subgroup? 
 
\probskip

%% 17 %%%%%%%%%%%%%%%%%%%%%%%%%%%%%%%%%%%%%%%%%%%%%%%%
\item[{\bf 17.}] 
Prove that $S_n$ is nonabelian for $n \geq 3$.

\probskip
 
%% 18 %%%%%%%%%%%%%%%%%%%%%%%%%%%%%%%%%%%%%%%%%%%%%%%%
\item[{\bf 18.}] 
Prove that $A_n$ is nonabelian for $n \geq 4$.

\probskip
 
%% 27 %%%%%%%%%%%%%%%%%%%%%%%%%%%%%%%%%%%%%%%%%%%%%%%%
\item[{\bf 27.}] 
Let $G$ be a group and define a map $\lambda_g : G \rightarrow G$ by
$\lambda_g(a) = g a$.  Prove that $\lambda_g$ is a permutation of $G$.

\newpage

%   Additional suggested exercises: 29, 31, 32, 33.  \\
\noindent {\bf Additional suggested exercises:} 29, 31, 32, 33.  
\\

\item[{\bf 29.}]  % 29
Recall that the \boldemph{center}\index{Group!center of} of a group $G$ is
\[
Z(G) = \{ g \in G : \mbox{$gx = xg$ for all $x \in G$} \}.
\]
Find the center of $D_8$. What about the center of $D_{10}$? What is
the center of $D_n$? 
 
\probskip
 
 
\item[{\bf 31.}]  % 31
For $\alpha$ and $\beta$ in $S_n$, define $\alpha \sim \beta$ if there
exists an $\sigma \in S_n$ such that 
$\sigma \alpha \sigma^{-1} = \beta$.  Show that $\sim$ is an equivalence
relation on $S_n$. 
 
\probskip
 
 
\item[{\bf 32.}]  % 32
Let $\sigma \in S_X$. If $\sigma^n(x) = y$, we will say that 
$x \sim y$. 
\begin{enumerate}
 
 \item
Show that $\sim$ is an equivalence relation on $X$.
 
 \item
If $\sigma \in A_n$ and $\tau \in S_n$, show that 
$\tau^{-1} \sigma \tau \in A_n$. 
 
\item
Define the \boldemph{orbit}\index{Orbit} of $x \in X$ under $\sigma \in S_X$ to
be the set  
\[
{\mathcal O}_{x, \sigma} = \{ y : x \sim y  \}.
\]
Compute the orbits of $\alpha, \beta, \gamma$ where
\begin{align*}
\alpha & = (1254) \\
\beta & = (123)(45)\\
\gamma & = (13)(25).
\end{align*}
 
 \item
If ${\mathcal O}_{x, \sigma} \cap {\mathcal O}_{y, \sigma} \neq \emptyset$,
prove that ${\mathcal O}_{x, \sigma} = {\mathcal O}_{y, \sigma}$.  The orbits
under a permutation $\sigma$ are the equivalence classes corresponding
to the equivalence relation $\sim$.
 
 
\item
A subgroup $H$ of $S_X$ is \boldemph{
transitive}\index{Subgroup!transitive} if for every $x, y \in X$, 
there exists a $\sigma \in H$ such that $\sigma(x) =y$. Prove that
$\langle \sigma \rangle$ is transitive if and only if 
${\mathcal O}_{x, \sigma} = X$ for some $x \in X$. 
 
 
\end{enumerate}
 
\probskip
 
 
\item[{\bf 33.}]  %33
Let $\alpha \in S_n$ for $n \geq 3$. If $\alpha \beta = \beta \alpha$
for all $\beta \in S_n$, prove that $\alpha$ must be the identity
permutation; hence, the center of $S_n$ is the trivial subgroup. 
 
 
\end{enumerate}
\end{document}
