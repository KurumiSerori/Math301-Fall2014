\documentclass[12pt,reqno]{amsart}
\usepackage[top=2cm, left=2cm,right=2cm,bottom=2cm]{geometry}
\renewcommand{\baselinestretch}{1.2}
\usepackage{amsmath}
\usepackage{amssymb}
\usepackage{tikz}
\usepackage{color,hyperref,enumerate,multicol}
\definecolor{darkblue}{rgb}{0.0,0.0,0.3}
\hypersetup{colorlinks,breaklinks,
            linkcolor=darkblue,urlcolor=darkblue,
            anchorcolor=darkblue,citecolor=darkblue}
            
\usepackage{algorithm}
\usepackage{algorithmic}
\pagestyle{empty}
\newcommand{\N}{\ensuremath{\mathbb{N}}}
\newcommand{\Z}{\ensuremath{\mathbb{Z}}}
\newcommand{\R}{\ensuremath{\mathbb{R}}}
\newcommand{\meet}{\ensuremath{\wedge}}
\newcommand{\Meet}{\ensuremath{\bigwedge}}
\newcommand{\join}{\ensuremath{\vee}}
\renewcommand{\emptyset}{\ensuremath{\varnothing}}
\renewcommand{\subset}{\ensuremath{\subsetneq}}
\newcommand{\boldemph}{\emph}
\newcommand{\lcm}{\ensuremath{\operatorname{lcm}}}
\newcommand{\lub}{\ensuremath{\operatorname{lub}}}
\newcommand{\glb}{\ensuremath{\operatorname{glb}}}
\newcommand{\Sub}{\ensuremath{\operatorname{Sub}}}
\renewcommand{\>}{\ensuremath{\rangle}}
\newcommand{\<}{\ensuremath{\langle}}

\newcommand{\probskip}{\vskip1cm}

\begin{document}
\thispagestyle{empty}

\noindent \textbf{Math 301} \hskip4cm {\bf Homework 7 -- Solutions} \hfill {\bf Fall 2014}
\vskip1cm
\noindent {\bf Exercises:} 1, 2 (below) and Judson 19.3, 19.14, 19.20.\\
{\bf Due date:} Wednesday, 10/22

\bigskip

\begin{enumerate}[{\bf 1.}]

%% 1 %%%%%%%%%%%%%%%%%%%%%%%%%%%%%%%%%%%%%%%%%%%%%%%%
\item %1
Let $P$ with $\leq$ be a partially ordered set, let $S \subseteq P$ and let
$u\in P$.  We say that $u$ is an \emph{upper bound} for $S$ iff $s\leq u$ for
all $s \in S$.  We say $\ell$ is the \emph{least upper bound} of $S$ iff $\ell$
is an upper bound of $S$ and $\ell \leq u$ for every upper bound $u$ of $S$.
Prove that if $\ell$ is the least upper bound of the set $\{x, y\}$ and $m$ is
the least upper bound of the set $\{\ell, z\}$, then $m$ is the least upper
bound of the set $\{x, y, z\}$.

\bigskip
\noindent
{\bf Solution:}
Since $\ell = \lub \{x, y\}$ and $m= \lub \{\ell, z\}$, we have 
\[
x\leq \ell, \qquad y \leq \ell  \quad  \text{ and } \quad \ell \leq m, \qquad z \leq m.
\]
Therefore, $x\leq \ell \leq m$, so $x\leq m$, by transitivity of $\leq$.
Similarly, $y\leq \ell \leq m$, so
transitivity implies $y\leq m$.  It follows that $m$ is an upper bound of the set
$\{x, y, z\}$.  We want to show that $m$ is the \emph{least} upper bound of 
$\{x, y, z\}$. That is, if $n$ is another upper bound of 
$\{x, y, z\}$, we must show $m\leq n$.

If $n$ is an upper bound of $\{x, y, z\}$, then it is also an upper bound of
$\{x, y\}$, and since $\ell$ is the least upper bound of $\{x,y\}$, we have
$\ell \leq n$.  Therefore, $n$ is an upper bound of $\{\ell, z\}$. Since $m$ is
the least upper bound of $\{\ell, z\}$, we have $m\leq n$.
\qed

\probskip

%% 2 %%%%%%%%%%%%%%%%%%%%%%%%%%%%%%%%%%%%%%%%%%%%%%%%
\item
Let $(P, \leq)$ be a partially ordered set with the property that every pair of
elements $x, y \in P$ has a greatest lower bound. For $x, y\in P$, define 
$x \cdot y = \operatorname{glb}(x,y)$. Prove that $(P, \cdot)$ is a semilattice.

%% Let S with · be a semilattice. For x, y ∈ S we say x ≤ y iff x · y = y. Prove that ≤ is a partial ordering on S. Also prove that x · y is the least upper bound of the set {x, y}.

\bigskip
\noindent
{\bf Solution:}
Recall, a \emph{semilattice} is a commutative idempotent semigroup, so we must
check that $\<P, \cdot\>$ has these three properties.
\begin{enumerate}[(i)]
\item To show $\<P, \cdot\>$ is a semigroup, we must prove that $\cdot$ is
  associative. Indeed, for all $x, y, z\in P$, we have
  \begin{align*}
    x\cdot (y \cdot z) &= \glb\{x, \glb\{y, z\}\}\\
    &= \glb\{x, y, z\} \qquad \qquad \text{(by Problem 1)}\\
    &= \glb\{\glb\{x,y\}, z\} \quad \quad \text{(by Problem 1)}\\
    &= (x\cdot y) \cdot z.
  \end{align*}

\item To prove $\cdot$ is commutative, note that the set $\{x,y\}$ is the same
  as the set $\{y,x\}$, so $x\cdot y = \glb\{x,y\} =\glb\{y,x\} = y\cdot x$.

\item Recall the following definition from lecture: an operation 
  $f: A^n \rightarrow A$ is called \emph{idempotent} if it satisfies 
  $f(a, \dots, a)  = a$ for all $a\in A$.  
  Thus, to say that $\cdot$ is an idempotent operation is to say that
  $a\cdot a = a$, for all $a\in P$. Indeed, $a\cdot a = \glb\{a, a\} = a$. 
\end{enumerate}

\probskip
 
%% 3 %%%%%%%%%%%%%%%%%%%%%%%%%%%%%%%%%%%%%%%%%%%%%%%%
\item[{\bf 19.3.}] 
Draw a diagram of the lattice of subgroups of ${\mathbb Z}_{12}$.
 
 
\bigskip
\noindent
{\bf Solution:}
How one determines that the diagram shown below represents the lattice of
subgroups of $\Z_{12}$ was explained in lecture.  
If it is not clear to you, please attend office hours, or post a question on
the class wiki, or email the instructor, or ask about it in lecture.
\begin{center}
\begin{tikzpicture}[scale=0.75]
  \node (1) at (0,0) {$\langle e \rangle$}; 
  \node (6) at (2,2) {$\langle 6 \rangle$}; 
  \node (4) at (-2,2) {$\langle 4 \rangle$}; 
  \node (2) at (0,4) {$\langle 2 \rangle$}; 
  \node (3) at (4,4) {$\langle 3 \rangle$}; 
  \node (G) at (2,6) {$\Z_{12}$}; 

  \draw (1) to (6) to (2) to (4) to (1);
  \draw (6) to (3) to (G) to (2) to (4);

\end{tikzpicture}
\end{center}

\probskip

%% 14 %%%%%%%%%%%%%%%%%%%%%%%%%%%%%%%%%%%%%%%%%%%%%%%%
\item[{\bf 19.14.}] 
Let $G$ be a group and $X$ be the set of subgroups of $G$ ordered by
set-theoretic inclusion. If $H$ and $K$ are subgroups of $G$, show
that the least upper bound of $H$ and $K$ is the subgroup generated by
$H \cup K$. 
 
\bigskip
\noindent
{\bf Solution:} This question is about the poset $\< \Sub(G), \leq\>$, where the
universe is the set $\Sub(G)$ of all subgroups of $G$, and the partial order is
the relation $\leq$ defined as follows: 
\[
H \leq K \qquad \Longleftrightarrow \qquad \text{$H$ is a subgroup of $K$.}
\]  

Let $X$ denote the least upper
bound of $H$ and $K$ in $\< \Sub(G), \leq\>$. Then, since $X$ belongs to
$\Sub(G)$, it is a subgroup of $G$, and since $X$  is an upper bound of $H$
and of $K$, we have $H\leq X$ and $K \leq X$.  Therefore, all elements of 
$H\cup K$ are contained in $X$.  Let $Y$ be any other subgroup of $G$
that contains $H \cup K$, then $H\leq Y$ and $K \leq Y$. Since $X$ is the
\emph{least} upper bound of $H$ and $K$, we have $X \leq Y$.  What we have shown
is that $X$ is the smallest subgroup of $G$ that contains the set $H\cup K$.
This is the definition of \emph{the subgroup of $G$ generated by} $H\cup K$.
\qed
\probskip
 
%% 20 %%%%%%%%%%%%%%%%%%%%%%%%%%%%%%%%%%%%%%%%%%%%%%%%
\item[{\bf 19.20.}] 
Let $X$ and $Y$ be posets.  A map $\phi : X \rightarrow Y$ is \boldemph{
order-preserving} if $a \preceq b$
implies that $\phi(a) \preceq \phi(b)$.  Let $L$ and $M$ be lattices.
A map $\psi: L \rightarrow M$ is a \boldemph{lattice
homomorphism}
if $\psi( a \vee b ) = \psi(a) \vee \psi(b)$ and $\psi( a \wedge b ) =
\psi(a) \wedge \psi(b)$. Show that every lattice homomorphism is
order-preserving, but that it is not the case that every
order-preserving map is a lattice homomorphism.  
 
\bigskip
\noindent
{\bf Solution:}
Let $\psi: L \rightarrow M$ be a lattice homomorphism.  We must show that $\psi$
is order preserving.  Let $a, b\in L$ be such that $a\leq b$.  As we learned in
lecture, if $a$ and $b$ have a greatest lower bound (as they must in a lattice),
then the statement $a\leq b$ is equivalent to $a\meet b = a$. Therefore,  
\begin{equation*}
%  \label{eq:1}
\psi(a) = \psi(a \meet b)  = \psi(a) \meet \psi(b).
\end{equation*}
(The second equality holds since $\psi$ is a lattice homomorphism.)
As mentioned, the equation $\psi(a) = \psi(a) \meet \psi(b)$ 
is equivalent to $\psi(a) \leq \psi(b)$.  We have thus proved that if $\psi$ is
a lattice homomorphism and if $a\leq b$, then $\psi(a) \leq \psi(b)$.  That is,
lattice homomorphisms are order preserving.

The converse is false.  That is, an order preserving map need not be a lattice
homomorphism. For example, consider the two lattice diagrams in the figure
below.  The map 
\[
\phi: \{0, 1, 2, 3\} \rightarrow \{a, b, c\} \quad \text{ defined by }
\]
\[
\phi(0) = a, \quad \phi(2) = b = \phi(3), \quad \phi(1) = c
\]
is clearly order preserving, but it is not a lattice homomorphism.  For example, 
\[
a = \phi(0) = \phi(2 \meet 3) \neq \phi(2) \meet \phi(3)  = b \meet b = b.
\]

\vskip5mm

\begin{center}
  
\begin{tikzpicture}[scale=0.7]
  \node (0) at (0,0) [draw, inner sep=1pt] {};
  \node (1) at (1.5,2)[draw,circle,inner sep=1pt] {}; 
  \node (2) at (-1.5,2) [draw,circle,inner sep=1pt] {};
  \node (3) at (0,4) [draw,circle,inner sep=1pt] {};
  \node (a) at (4,0) [draw,circle,inner sep=1pt] {};
  \node (b) at (4,2) [draw,circle,inner sep=1pt] {};
  \node (c) at (4,4) [draw,circle,inner sep=1pt] {};
  \draw (a) to (b) to (c);
  \draw (0) to (1) to (3) to (2) to (0);

  \draw (0) node [below] {$0$};
  \draw (2) node [left] {$2$};
  \draw (1) node [right] {$3$};
  \draw (3) node [above] {$1$};
  \draw (a) node [right] {$a$};
  \draw (b) node [right] {$b$};
  \draw (c) node [right] {$c$};
\end{tikzpicture}
\end{center}
\end{enumerate}
\end{document}
