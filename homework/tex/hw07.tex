\documentclass[12pt,reqno]{amsart}
\usepackage[top=2cm, left=2cm,right=2cm,bottom=2cm]{geometry}
\renewcommand{\baselinestretch}{1.2}
\usepackage{amsmath}
\usepackage{amssymb}
\usepackage{color,hyperref,enumerate,multicol}
\definecolor{darkblue}{rgb}{0.0,0.0,0.3}
\hypersetup{colorlinks,breaklinks,
            linkcolor=darkblue,urlcolor=darkblue,
            anchorcolor=darkblue,citecolor=darkblue}
            
\usepackage{algorithm}
\usepackage{algorithmic}
\pagestyle{empty}
\newcommand{\N}{\ensuremath{\mathbb{N}}}
\newcommand{\Z}{\ensuremath{\mathbb{Z}}}
\newcommand{\R}{\ensuremath{\mathbb{R}}}
\newcommand{\meet}{\ensuremath{\wedge}}
\newcommand{\Meet}{\ensuremath{\bigwedge}}
\newcommand{\join}{\ensuremath{\vee}}
\renewcommand{\emptyset}{\ensuremath{\varnothing}}
\renewcommand{\subset}{\ensuremath{\subsetneq}}
\newcommand{\boldemph}{\emph}
\newcommand{\lcm}{\operatorname{lcm}}

\newcommand{\probskip}{\vskip1cm}

\begin{document}
\thispagestyle{empty}

\noindent \textbf{Math 301} \hskip5cm {\bf Homework 7} \hfill {\bf Fall 2014}
\vskip1cm
\noindent {\bf Exercises:} 1, 2 (below) and Judson 19.3, 19.14, 19.20.\\
{\bf Due date:} Wednesday, 10/22

\bigskip

\begin{enumerate}[{\bf 1.}]

%% 1 %%%%%%%%%%%%%%%%%%%%%%%%%%%%%%%%%%%%%%%%%%%%%%%%
\item %1
Let $P$ with $\leq$ be a partially ordered set, let $S \subseteq P$ and let
$u\in P$.  We say that $u$ is an \emph{upper bound} for $S$ iff $s\leq u$ for
all $s \in S$.  We say $\ell$ is the \emph{least upper bound} of $S$ iff $\ell$
is an upper bound of $S$ and $\ell \leq u$ for every upper bound $u$ of $S$.
Prove that if $\ell$ is the least upper bound of the set $\{x, y\}$ and $m$ is
the least upper bound of the set $\{\ell, z\}$, then $m$ is the least upper
bound of the set $\{x, y, z\}$.

\probskip

%% 2 %%%%%%%%%%%%%%%%%%%%%%%%%%%%%%%%%%%%%%%%%%%%%%%%
\item
Let $S$ with $\cdot$ be a semilattice.  For $x, y\in S$ we say $x\leq y$ iff
$x\cdot y  = y$. Prove that $\leq$ is a partial ordering on $S$.  Also prove
that $x\cdot y$ is the least upper bound of the set $\{x,y\}$.

\probskip
 
%% 3 %%%%%%%%%%%%%%%%%%%%%%%%%%%%%%%%%%%%%%%%%%%%%%%%
\item[{\bf 19.3.}] 
Draw a diagram of the lattice of subgroups of ${\mathbb Z}_{12}$.
 
 
\probskip

%% 14 %%%%%%%%%%%%%%%%%%%%%%%%%%%%%%%%%%%%%%%%%%%%%%%%
\item[{\bf 19.14.}] 
Let $G$ be a group and $X$ be the set of subgroups of $G$ ordered by
set-theoretic inclusion. If $H$ and $K$ are subgroups of $G$, show
that the least upper bound of $H$ and $K$ is the subgroup generated by
$H \cup K$. 
 
\probskip
 
%% 20 %%%%%%%%%%%%%%%%%%%%%%%%%%%%%%%%%%%%%%%%%%%%%%%%
\item[{\bf 19.20.}] 
Let $X$ and $Y$ be posets.  A map $\phi : X \rightarrow Y$ is \boldemph{
order-preserving} if $a \preceq b$
implies that $\phi(a) \preceq \phi(b)$.  Let $L$ and $M$ be lattices.
A map $\psi: L \rightarrow M$ is a \boldemph{lattice
homomorphism}
if $\psi( a \vee b ) = \psi(a) \vee \psi(b)$ and $\psi( a \wedge b ) =
\psi(a) \wedge \psi(b)$. Show that every lattice homomorphism is
order-preserving, but that it is not the case that every
order-preserving homomorphism is a lattice homomorphism.  
 
 
 
\end{enumerate}
\end{document}
