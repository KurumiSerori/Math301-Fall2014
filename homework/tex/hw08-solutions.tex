\documentclass[12pt,reqno]{amsart}
\usepackage[top=2cm, left=2cm,right=2cm,bottom=2cm]{geometry}
\renewcommand{\baselinestretch}{1.2}
\usepackage{amsmath}
\usepackage{amssymb}
\usepackage{color,hyperref,enumerate,multicol}
\definecolor{darkblue}{rgb}{0.0,0.0,0.3}
\hypersetup{colorlinks,breaklinks,
            linkcolor=darkblue,urlcolor=darkblue,
            anchorcolor=darkblue,citecolor=darkblue}
\pagestyle{empty}
\newcommand{\N}{\ensuremath{\mathbb{N}}}
\newcommand{\Z}{\ensuremath{\mathbb{Z}}}
\newcommand{\R}{\ensuremath{\mathbb{R}}}
\newcommand{\meet}{\ensuremath{\wedge}}
\newcommand{\Meet}{\ensuremath{\bigwedge}}
\newcommand{\join}{\ensuremath{\vee}}
\renewcommand{\emptyset}{\ensuremath{\varnothing}}
\renewcommand{\subset}{\ensuremath{\subsetneq}}
\newcommand{\boldemph}{\emph}
\newcommand{\lcm}{\operatorname{lcm}}

%%% better divides symbol >>>>>>>>>
\makeatletter
\def\localbig#1#2{%
  \sbox\z@{$#1
    \sbox\tw@{$#1()$}%
    \dimen@=\ht\tw@\advance\dimen@\dp\tw@
    \nulldelimiterspace\z@\left#2\vcenter to1.2\dimen@{}\right.
  $}\box\z@}
\makeatletter
\newcommand{\divides}{\mathrel{\mathpalette\dividesaux\relax}}
\newcommand{\ndivides}{\mathrel{\mathpalette\ndividesaux\relax}}
\newcommand{\dividesaux}[2]{\mbox{$#1\localbig{#1}|$}}
\newcommand{\ndividesaux}[2]{%
  \mkern.5mu\ooalign{\hidewidth$#1\localbig{#1}|$\hidewidth\cr$#1\nmid$\cr}}
%%% <<<<<<<<<< better divides symbol

\newcommand{\<}{\ensuremath{\langle}}
\renewcommand{\>}{\ensuremath{\rangle}}

\newcommand{\probskip}{\vskip1cm}

\begin{document}
\thispagestyle{empty}

\noindent \textbf{Math 301} \hskip5cm {\bf Homework 8} \hfill {\bf Fall 2014}
\vskip1cm
\noindent {\bf Exercises:} 1 below and Judson: 6.5bd, 6.11ade, 6.16, 6.18\\
{\bf Due date:} Friday, 10/24

\bigskip


\begin{enumerate}[{\bf 1.}]

%% 1 %%%%%%%%%%%%%%%%%%%%%%%%%%%%%%%%%%%%%%%%%%%%%%%%
\item
Prove or disprove the following: 
\begin{enumerate}
\item 
There exists a group $G$ of order $|G| = 8$ with an element
$g \in G$ of order $|g|=3$.

\item
If $H$ and $K$ are subgroups of a group $G$ with $|H|=2$ and $|K|=3$, 
  then $|G|\geq 6$.

\item
Every subgroup of the integers has finite index.

\item 
Every subgroup of the integers has finite order.
\end{enumerate}

\medskip

\noindent {\bf Solution:} 
\begin{enumerate}
\item False.  Note that $|g| = |\<g\>|$, and by Lagrange's Theorem $|\<g\>|$
  divides $|G|$.  So the order of every element of a group must divide the
  order of the group. Since $3 \ndivides 8$, there is no element of order 3 in a
  group of order 8.
\medskip
\item True.  If $H, K \leq G$ then $|H| \divides |G|$ and
$|K| \divides |G|$, by Lagrange's Theorem. 
In case $|H| = 2$ and $|K| = 3$, we have
$2\divides |G|$ and $3\divides |G|$, 
so $|G| \geq \lcm(2,3) = 6$.
\medskip
\item
False. The subgroup $ \{0\} \leq \Z$ has infinite index, 
$[\Z : \{0\}] = \infty$. The cosets of $\{0\}$ in $\Z$ have the form
$k+\{0\} = \{k\}$, where $k\in \Z$.  That is,  for each $k\in \Z$, the set $\{k\}$ is the coset
of $\{0\}$ containing $k$.
%$\{0\}, 1+\{0\}, 2+\{0\}, \dots, , \dots$.
\medskip
\item
False. The subgroup $H = 2\Z \leq \Z$ has infinite order.
\end{enumerate}
\probskip

%% 5 %%%%%%%%%%%%%%%%%%%%%%%%%%%%%%%%%%%%%%%%%%%%%%%%
\item[{\bf 6.5.}]
In each case below, list the left cosets of $H$ in $G$.
\begin{enumerate}
\item[{\bf b.}]
$G = U(8)$, $H = \langle 3 \rangle$.
\item[{\bf c.}]
$G = S_4$, $H = A_4$.
\end{enumerate}

\medskip

\noindent {\bf Solution:} 
\begin{enumerate}
\item[{\bf b.}]
If $G = U(8) = \{1, 3, 5, 7\}$ and $H = \langle 3 \rangle = \{1, 3\}$, then
there are $[G:H] = |G|/|H| = 4/2 = 2$ cosets, namely, $H = \{1,3\}$ and 
$5H = \{5, 7\}$. That is, 
\[
G/H =  %% U(8)/\<3\> = 
\{gH : g\in G\} = \{ H, 5H\} = \{\{1,3\}, \{5, 7\}\}.
\]
\item[{\bf c.}] 
If $G = S_4$ and $H = A_4$, then there are $[G:H] = |G|/|H| = 4!/(4!/2) = 2$
cosets, namely, $A_4$ and $gA_4$, where $g$ is any element of $S_4$ that does
not belong to $A_4$. (For example, let $g = (12)$, which is odd.)
That is, $G/H %% = S_4/A_4 
= \{gH : g\in G\} = \{ A_4, (12)A_4\}$.

\end{enumerate}

\newpage
%% 11 %%%%%%%%%%%%%%%%%%%%%%%%%%%%%%%%%%%%%%%%%%%%%%%%
\item[{\bf 6.11.}] 
Let $H$ be a subgroup of a group $G$ and suppose that $g_1, g_2 \in G$.  Prove
that the following conditions are equivalent: 
\begin{enumerate}
 
\item[(a)]
$g_1 H = g_2 H$
 
\item[(d)]
$g_2 \in g_1 H$
 
\item[(e)]
$g_1^{-1} g_2 \in H$
\end{enumerate}

\medskip

\noindent {\bf Solution:} 
This was proved in class.  Please come to office hours if you don't understand it.

\probskip


 
%% 16 %%%%%%%%%%%%%%%%%%%%%%%%%%%%%%%%%%%%%%%%%%%%%%%%
\item[{\bf 6.16.}] 
If $|G| = 2n$, prove that the number of elements of order 2 is odd.  Use this
result to show that $G$ must contain a subgroup of order 2. 

\medskip

\noindent {\bf Solution:} 
Suppose $|G| = 2n$.  Let 
\[
X = \{x\in G | x^2 =e\} = \{x\in G | x =x^{-1}\},
\]
\[
X^c = \{x\in G | x^2 \neq e\} = \{x\in G | x \neq x^{-1}\}.
\]
Then $G$ is the disjoint union $G = X \coprod X^c$, so
\begin{equation}
|G| = |X| + |X^c|.
\label{eq:1}
\end{equation}
Note that $X$ includes the identity element, 
which has order 1, so $X \setminus \{e\}$ is the set of all elements of
$G$ of order~2. There are $|X|-1$ such elements, so the goal is to show that
$|X| -1$ is odd, or, equivalently, that $|X|$ is even.
Now, $|X^c|$ is clearly even since, for each $x \in X^c$, there is a corresponding element
$x^{-1}$ (distinct from $x$) that also belongs to $X^c$.
%%  (since $(x^{-1})^{-1} = x$, so $x^{-1} \neq (x^{-1})^{-1}$)
That is, elements of $X^c$ come in pairs.  Therefore, $|X^c|=2k$ for some $k$,
so by~(\ref{eq:1}),
\[
|X| = |G| - |X^c| = 2n - 2k = 2(n-k),
\]
which is an even number.
\qed
\probskip


%% 18 %%%%%%%%%%%%%%%%%%%%%%%%%%%%%%%%%%%%%%%%%%%%%%%%
\item[{\bf 6.18.}] 
If $[G : H] = 2$, prove that $gH = Hg$.

\noindent {\bf Solution:} 
If $g \in H$, then $gH = H = Hg$ and we are done.  So assume $g \notin H$.  
Then $H\neq gH$.  Since $[G : H] = 2$, there are exactly two left cosets of $H$
in $G$, namely  $H$ and $gH$.
Similarly, there are two right cosets of $H$ in $G$, namely $H$ and $Hg$.  
Since $G$ is the disjoint union of $H$ and $Hg$, and also the disjoint union of
$H$ and $gH$, we have $gH = Hg$.
\end{enumerate}
\end{document}

