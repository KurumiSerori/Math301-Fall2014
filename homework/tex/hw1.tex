%%%%(c)
%%%%(c)  This file is a portion of the source for the textbook
%%%%(c)
%%%%(c)    Abstract Algebra: Theory and Applications
%%%%(c)    Copyright 1997 by Thomas W. Judson
%%%%(c)
%%%%(c)  See the file COPYING.txt for copying conditions
%%%%(c)
%%%%(c)
 
\markright{EXERCISES}
\section*{Exercises}
\exrule
 
{\small
\begin{enumerate}

% 1
\item
Suppose that
\begin{align*}
A & = \{ x : x \in \mathbb N \text{ and } x \text{ is even} \}, \\
B & = \{x : x \in \mathbb N \text{ and } x \text{ is prime}\}, \\
C & = \{ x : x \in \mathbb N \text{ and } x \text{ is a multiple of 5}\}.
\end{align*}
Describe each of the following sets. 
\begin{multicols}{2}
\begin{enumerate}

\item
$A \cap B$

\item
$B \cap C$

\item
$A \cup B$

\item
$A \cap (B \cup C)$

\end{enumerate}
\end{multicols}
 
% 2
\item
If $A = \{ a, b, c \}$, $B = \{ 1, 2, 3 \}$, $C = \{ x \}$, and $D = \emptyset$, list all of the elements in each of the following sets. 
\begin{multicols}{2}
\begin{enumerate}

\item
$A \times B$

\item
$B \times A$

\item
$A \times B \times C$

\item
$A \times D$

\end{enumerate}
\end{multicols}
  
% 3
\item
Find an example of two nonempty sets $A$ and $B$ for which $A \times B = B \times A$ is true. 

% 4  
\item
Prove $A \cup \emptyset = A$ and $A \cap \emptyset = \emptyset$.
 
% 5
\item
Prove $A \cup B = B \cup A$ and $A \cap B = B \cap A$.
 
% 6
\item
Prove $A \cup (B \cap C) = (A \cup B) \cap (A \cup C)$.
 
% 7
\item
Prove $A \cap (B \cup C) = (A \cap B) \cup (A \cap C)$.
 
% 8
\item
Prove  $A \subset B$ if and only if $A \cap B = A$.
 
% 9
\item
Prove $(A \cap B)' = A' \cup B'$.

% 10
\item
Prove  $A \cup B = (A \cap B) \cup (A \setminus B) \cup (B \setminus A)$. 
 
% 11
\item
Prove  $(A \cup B) \times C = (A \times C ) \cup (B \times C)$.
 
% 12
\item
Prove  $(A \cap B) \setminus B = \emptyset$.
 
% 13
\item
Prove  $(A \cup B) \setminus B = A \setminus B$.
 
% 14
\item
Prove  $A \setminus (B \cup C) = (A \setminus B) \cap (A \setminus C)$. 

% 15
\item
Prove  $A \cap (B \setminus C) = (A \cap B) \setminus (A \cap C)$. 
 
 
%% TWJ, 2010/03/31
%% Fixed the error in the exercise
% 16
\item
Prove  $(A \setminus B) \cup (B \setminus A) = (A \cup B) \setminus (A \cap B)$. 
 
% 17
\item
Which of the following relations $f: {\mathbb Q} \rightarrow {\mathbb Q}$
define a mapping? In each case, supply a reason why $f$ is or is not a
mapping. 
\begin{multicols}{2}
\begin{enumerate}

\item 
$\displaystyle f(p/q) = \frac{p+ 1}{p - 2}$

\item 
$\displaystyle f(p/q) = \frac{3p}{3q}$

\item 
$\displaystyle f(p/q) = \frac{p+q}{q^2}$

\item 
$\displaystyle f(p/q) = \frac{3 p^2}{7 q^2} - \frac{p}{q}$

\end{enumerate}
\end{multicols}
 
% 18
\item
Determine which of the following functions are one-to-one and which are onto.  If the function is not onto, determine its range.
\begin{enumerate} 
 
\item
$f: {\mathbb R} \rightarrow {\mathbb R}$ defined by $f(x) = e^x$
 
\item
$f: {\mathbb Z} \rightarrow {\mathbb Z}$ defined by $f(n) = n^2 + 3$
  
\item
$f: {\mathbb R} \rightarrow {\mathbb R}$ defined by $f(x) = \sin x$
 
\item
$f: {\mathbb Z} \rightarrow {\mathbb Z}$ defined by $f(x) = x^2$
 
\end{enumerate}
 
% 19
\item
Let $f :A \rightarrow B$ and $g : B \rightarrow C$ be invertible mappings; that is, mappings such that $f^{-1}$ and $g^{-1}$ exist.  Show that $(g \circ f)^{-1} =f^{-1} \circ g^{-1}$. 

% 20
\item
\begin{enumerate}
  
\item
Define a function $f: {\mathbb N} \rightarrow {\mathbb N}$ that is one-to-one but not onto. 
 
\item
Define a function $f: {\mathbb N} \rightarrow {\mathbb N}$ that is onto but not one-to-one. 
 
\end{enumerate}
 
% 21
\item
Prove the relation defined on ${\mathbb R}^2$ by $(x_1, y_1 ) \sim (x_2, y_2)$ if $x_1^2 + y_1^2 = x_2^2 + y_2^2$ is  an equivalence relation. 
 
% 22
\item
Let $f : A \rightarrow B$ and $g : B \rightarrow C$ be maps.
\begin{enumerate}
 
\item
If $f$ and $g$ are both one-to-one functions, show that $g \circ f$
is one-to-one. 
 
\item
If $g \circ f$ is onto, show that $g$ is onto.
 
\item
If $g \circ f$ is one-to-one, show that $f$ is one-to-one.
 
\item
If $g \circ f$ is one-to-one and $f$ is onto, show that $g$ is
one-to-one.
 
\item
If $g \circ f$ is onto and $g$ is one-to-one, show that $f$ is onto.
 
\end{enumerate}
 
% 23
\item
Define a function on the real numbers by
\[
f(x) = \frac{x + 1}{x - 1}.
\]
What are the domain and range of $f$? What is the inverse of $f$?  Compute $f \circ f^{-1}$ and $f^{-1} \circ f$. 
 
% 24
\item
Let $f: X \rightarrow Y$ be a map with $A_1, A_2 \subset X$ and $B_1, B_2 \subset Y$. 
\begin{enumerate}
 
\item
Prove $f( A_1 \cup A_2 ) = f( A_1) \cup f( A_2 )$.
 
\item
Prove $f( A_1 \cap A_2 ) \subset f( A_1) \cap f( A_2 )$.  Give an example in which equality fails.
 
\item
Prove $f^{-1}( B_1 \cup B_2 ) = f^{-1}( B_1) \cup f^{-1}(B_2 )$, where
\[
f^{-1}(B) = \{ x \in X : f(x) \in B \}.
\]
 
\item
Prove $f^{-1}( B_1 \cap B_2 ) = f^{-1}( B_1) \cap f^{-1}( B_2 )$. 
 
\item
Prove $f^{-1}( Y \setminus B_1 ) = X \setminus f^{-1}( B_1)$.
 
\end{enumerate}
 
% 25
\item
Determine whether or not the following relations are equivalence relations on the given set.  If the relation is an equivalence relation, describe the partition given by it.  If the relation is not an equivalence relation, state why it fails to be one.
\begin{multicols}{2}
\begin{enumerate}
 
\item
$x \sim y$ in ${\mathbb R}$ if $x \geq y$
 
\item
$m \sim n$ in ${\mathbb Z}$ if $mn > 0$
 
\item
$x \sim y$ in ${\mathbb R}$ if $|x - y| \leq 4$
 
\item
$m \sim n$ in ${\mathbb Z}$ if $m \equiv n \pmod{6}$
 
\end{enumerate}
\end{multicols}
 
 
\item
Define a relation $\sim$ on ${\mathbb R}^2$ by stating that $(a, b) \sim (c, d)$ if and only if $a^2 + b^2 \leq c^2 + d^2$. Show that $\sim$ is reflexive and transitive but not symmetric.
 
\item
Show that an $m \times n$ matrix gives rise to a well-defined  map from ${\mathbb R}^n$ to ${\mathbb R}^m$. 
 
\item
Find the error in the following argument by providing a counterexample. ``The reflexive property is redundant in the axioms for an equivalence relation.  If $x \sim y$, then $y \sim x$ by the symmetric property.  Using the transitive property, we can deduce that $x \sim x$.'' 
 
\item
\textbf{Projective Real Line.}
Define a relation on ${\mathbb R}^2 \setminus  (0,0)$ by letting $(x_1, y_1) \sim (x_2, y_2)$ if there exists a nonzero real number $\lambda$ such that $(x_1, y_1)  = ( \lambda x_2, \lambda y_2)$.  Prove that $\sim$ defines an equivalence relation on ${\mathbb R}^2 \setminus (0,0)$.  What are the corresponding  equivalence classes?  This equivalence relation defines the projective line, denoted by  ${\mathbb P}({\mathbb R} )$, which is very important in geometry.
 
\end{enumerate}
}
 
