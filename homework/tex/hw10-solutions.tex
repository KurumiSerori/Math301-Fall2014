\documentclass[12pt,reqno]{amsart}
\usepackage[top=2cm, left=2cm,right=2cm,bottom=2cm]{geometry}
\renewcommand{\baselinestretch}{1.2}
\usepackage{amsmath}
\usepackage{amssymb}
\usepackage{scalefnt}
\usepackage{tikz}
\usepackage{color,hyperref,enumerate,multicol}
\definecolor{darkblue}{rgb}{0.0,0.0,0.4}
\hypersetup{colorlinks,breaklinks,
            linkcolor=darkblue,urlcolor=darkblue,
            anchorcolor=darkblue,citecolor=darkblue}
            
\usepackage{algorithm}
\usepackage{algorithmic}
\pagestyle{empty}
\newcommand{\N}{\ensuremath{\mathbb{N}}}
\newcommand{\Z}{\ensuremath{\mathbb{Z}}}
\newcommand{\R}{\ensuremath{\mathbb{R}}}
\newcommand{\bL}{\ensuremath{\mathbf{L}}}
\newcommand{\bP}{\ensuremath{\mathbf{P}}}
\newcommand{\bQ}{\ensuremath{\mathbf{Q}}}
\newcommand{\bA}{\ensuremath{\mathbf{A}}}
\newcommand{\bB}{\ensuremath{\mathbf{B}}}
\newcommand{\bG}{\ensuremath{\mathbf{G}}}
\newcommand{\bH}{\ensuremath{\mathbf{H}}}
\newcommand{\invG}{\ensuremath{\operatorname{inv}^{\bG}}}
\newcommand{\invH}{\ensuremath{\operatorname{inv}^{\bH}}}
\newcommand{\meet}{\ensuremath{\wedge}}
\newcommand{\Meet}{\ensuremath{\bigwedge}}
\newcommand{\<}{\ensuremath{\langle}}
\renewcommand{\>}{\ensuremath{\rangle}}
\newcommand{\join}{\ensuremath{\vee}}
\renewcommand{\emptyset}{\ensuremath{\varnothing}}
\renewcommand{\subset}{\ensuremath{\subsetneq}}
\newcommand{\boldemph}{\emph}
\newcommand{\lcm}{\ensuremath{\operatorname{lcm}}}
\newcommand{\Sym}{\ensuremath{\operatorname{Sym}}}
%\newcommand{\bG}{\ensuremath{\mathbf{G}}}

\newcommand{\probskip}{\vskip1cm}

\begin{document}
\thispagestyle{empty}

\noindent \textbf{Math 301} \hskip5cm {\bf Homework 10} \hfill {\bf Fall 2014}
\vskip1cm
\noindent {\bf Exercises:} Judson 10.1abe, 10.5, 10.10, 10.11, 10.13acd, and
Problem 6 below.\\
{\bf Due date:} Wednesday, 11/05

\bigskip
\begin{enumerate}
%% 1 %%%%%%%%%%%%%%%%%%%%%%%%%%%%%%%%%%%%%%%%%%%%%%%%
\item[{\bf 10.1}] For each of the following groups $G$, determine whether $H$ is a normal
subgroup of $G$. If $H$ is a normal subgroup, write out a Cayley table
for the factor group $G/H$.
\begin{enumerate}
\item[(a)]
$G = S_4$ and $H = A_4$
 \item[(b)]
$G = A_5$ and $H = \{ (1), (123), (132) \}$
%%  \item
%% $G = S_4$ and $H = D_4$
%%  \item
%% $G = Q_8$ and $H = \{ 1, -1, I, -I \}$
\item[(e)]

$G = {\mathbb Z}$ and $H = 5 {\mathbb Z}$
 
\end{enumerate}

\medskip
\noindent {\bf Solution:} 
\begin{enumerate}
\item[(a)]
The subgroup $H = A_4$ has index $[S_4: A_4] = 2$.  Therefore, by Exercise 10.10
(below), $A_4$ is normal in $S_4$.  The elements of the factor group, that is,
the cosets of $A_4$ in $S_4$, are $\{A_4, gA_4\}$, where $g$ is any element of
$S_4$ that is not contained in $A_4$.  For example, $g = (23)$ works.  

Recall that, for a normal subgroup $N\triangleleft G$, coset
multiplication is defined by
$g_1N * g_2N = (g_1\cdot g_2) N$, where $g_1\cdot g_2$ is the product in $G$.
So one acceptable representation of the Cayley table of $S_4/A_4$ is

\medskip
\begin{center}
\begin{tabular}{r|rr}
  $*$ & $A_4$ &$(23)A_4$ \\
\hline
  $A_4$ & $A_4$ &$(23)A_4$ \\
  $(23)A_4$ & $(23)A_4$ &$A_4$
\end{tabular}
\end{center}
\medskip

An acceptable alternative is
\medskip
%% \begin{multicols}{3}
%% \end{multicols}
\begin{center}
\begin{tabular}{r|rr}
  $*$ & $A_4$ &$(12)A_4$ \\
\hline
  $A_4$ & $A_4$ &$(12)A_4$ \\
  $(12)A_4$ & $(12)A_4$ &$A_4$
\end{tabular}
\hskip1cm or even \hskip1cm
\begin{tabular}{r|rr}
  $*$ & $(123)A_4$ &$(34)A_4$ \\
\hline
  $(123)A_4$ & $(123)A_4$ &$(34)A_4$ \\
  $(34)A_4$ & $(34)A_4$ &$(123)A_4$
\end{tabular}
\end{center}
\medskip
or any other table involving two cosets $g_0A_4$ and $g_1A_4$, where
$g_0\in A_4$, so $g_0A_4 = A_4$ is the identity element, and 
$g\in S_4 - A_4$, so $g_1A_4 \neq A_4$ is the nonidentity element.

Note that $(123) \in A_4$ since it can be written as $(123) = (13)(12)$, which
is a product of an \emph{even} number of transpositions.  Therefore, $(123)A_4 = A_4$. For
this reason, we could have use $(123)A_4$ to represent the identity element of
the factor group.  Of course, we cannot use $(123)A_4$ as the nonidentity element,
so the following table would be incorrect:
\medskip
\begin{center}
\begin{tabular}{r|rr}
  $*$ & $A_4$ &$(123)A_4$ \\
\hline
  $A_4$ & $A_4$ & $(123)A_4$ \\
  $(123)A_4$ & $(123)A_4$ &$A_4$
\end{tabular}
\hskip1cm $\longleftarrow$ (not the Cayley table of $S_4/A_4$)
\end{center}

\newpage

 \item[(b)]
The subgroup $H = \{ (1), (123), (132) \}$ is not normal in $G = A_5$, as we
will show using\\[4pt] 
\indent {\bf a standard way to prove a subgroup $H$ is not normal in
  $G$:}
\medskip
\begin{quote}
\emph{find elements $g\in G$ and $h\in H$ such that $ghg^{-1} \notin H$.}
\end{quote}
\medskip

In the present example, if we let $g = (234) \in A_5$ and $h = (123) \in H$,
then 
\[
ghg^{-1}  = (234)(123)(243) = (234)(124) = (134) \notin H.
\]
\item[(e)]
Certainly $H = 5 {\mathbb Z}$ is normal in $G = {\mathbb Z}$, since $G$ is
abelian (so every subgroup of $G$ is normal).  The elements of the factor group
are the cosets of $5\Z$ in $\Z$, and the Cayley table can be presented as follows:
\medskip
\begin{center}
\begin{tabular}{r|rrrrr}
  $+$ & $5\Z$ &$1+5\Z$&$2+5\Z$&$3+5\Z$&$4+5\Z$ \\
\hline
  $5\Z$ & $5\Z$ &$1+5\Z$&$2+5\Z$&$3+5\Z$&$4+5\Z$ \\
  $1+5\Z$ &$1+5\Z$&$2+5\Z$&$3+5\Z$&$4+5\Z$& $5\Z$  \\
  $2+5\Z$ &$2+5\Z$&$3+5\Z$&$4+5\Z$& $5\Z$&$1+5\Z$  \\
  $3+5\Z$ &$3+5\Z$&$4+5\Z$& $5\Z$&$1+5\Z$&$2+5\Z$  \\
  $4+5\Z$ &$4+5\Z$& $5\Z$&$1+5\Z$&$2+5\Z$&$3+5\Z$
\end{tabular}
\end{center}
\medskip

It is also acceptable to use the shorthand $[k]$ or $(k)$ for the coset of $5\Z$
containing $k$, in which case, the Cayley table could be presented as follows:

\medskip
\begin{center}
\begin{tabular}{r|rrrrr}
  $+$ & $[0]$ &$[1] $&$[2] $&$[3] $&$[4] $ \\
\hline
  $[0]$ & $[0]$ &$[1] $&$[2] $&$[3] $&$[4] $ \\
  $[1] $ &$[1] $&$[2] $&$[3] $&$[4] $& $[0]$  \\
  $[2] $ &$[2] $&$[3] $&$[4] $& $[0]$&$[1] $  \\
  $[3] $ &$[3] $&$[4] $& $[0]$&$[1] $&$[2] $  \\
  $[4] $ &$[4] $& $[0]$&$[1] $&$[2] $&$[3] $
\end{tabular}
\end{center}
\medskip
which looks an awful lot like the group of integers with addition modulo 5 that
we encountered earlier, and called $\Z_5$.  In fact,  the group 
$\Z_5$, whose universe is the set of integers $\{0, 1, 2, 3, 4\}$ and whose
binary operation is addition modulo 5 is isomorphic to the group $\Z/5\Z$.  
While the elements of $\Z/5\Z$ are infinite sets of integers, the elements of $\Z_5$
are just the five integers $\{0, 1, 2, 3, 4\}$.  Apart from this distinction, 
the group structure is the same in each case, as we can see from the Cayley
tables.

 
\end{enumerate}

\bigskip

\item[{\bf 10.5.}]
Show that the intersection of two normal subgroups is a normal
subgroup. 
 
\medskip
\noindent {\bf Solution:} Let $H$ and $K$ be normal subgroups of a group $G$.
We have proved in the past that the intersection $N = H\cap K$ of two subgroups
is a subgroup.  We will now prove that $N$ is normal using\\[4pt]
\indent {\bf a standard way to prove a subgroup $N$ is normal in $G$:}
\medskip
\begin{quote}
\emph{Pick arbitrary elements $g\in G$ and $n\in N$ and show that $gng^{-1} \in N$.}
\end{quote}
\medskip
Fix $g\in G$ and $n \in N = H\cap K$.  
Since $n \in H$ and $H \triangleleft G$, we have $gng^{-1} \in H$.
Since $n \in K$ and $K \triangleleft G$, we have $gng^{-1} \in K$.
Therefore, $gng^{-1} \in H\cap K = N$.
\qed
\bigskip

\item[{\bf 10.10.}]
Let $H$ be a subgroup of index 2 of a group $G$. Prove that $H$ must
be a normal subgroup of $G$. Conclude that $S_n$ is not simple for $n \geq 3$.

\medskip
\noindent {\bf Solution:} We will show that $[G:H]=2$ implies $H\triangleleft G$
using\\[4pt]
\indent {\bf another standard way to prove a subgroup $H$ is normal in $G$:}
\medskip
\begin{quote}
\emph{Pick an arbitrary element $g\in G$ and show that $gH = Hg$.}
\end{quote}
\medskip
If $[G:H]=2$, then there are two left cosets of $H$ in $G$.
Fix $g \in G$. If $g \in H$, then $gH = Hg$ and there is nothing to prove.
Assume $g \notin H$.  Then the two left cosets of $H$ in $G$, 
are $H$ and $gH$. Recall that a full set of left cosets partitions the group as a
\emph{disjoint} union $G = H \cup gH$.  Similarly, the two right cosets of $H$ in
$G$ must be $H$ and $Hg$, and again we have a partition of $G$ as a into
\emph{disjoint} union of sets $G = H \cup Hg$.  It follows that $gH = G-H = Hg$.
\qed

\bigskip

\item[{\bf 10.11.}]
If a group $G$ has exactly one subgroup $H$ of order $k$, prove that
$H$ is normal in $G$. 

\medskip
\noindent {\bf Solution:} 
We will solve this using\\[4pt]
\indent {\bf another standard way to prove a subgroup $H$ is normal in $G$:}
\medskip
\begin{quote}
\emph{Pick an arbitrary element $g\in G$ and show that $gHg^{-1} = H$.}
\end{quote}
\medskip
First, given a subgroup $H\leq G$, and an arbitrary element $g\in G$, it is not hard to
see that the \emph{conjugate of $H$ by $g$}, which is defined by
\[
gHg^{-1} := \{ghg^{-1} | h\in H\},
\]
is also a subgroup of $G$.  Moreover, the function $h \mapsto ghg^{-1}$ is a
bijection.\footnote{In fact, as we will see later, $x\mapsto gxg^{-1}$ 
is an automorphism.}  
Therefore, $|H| = |gHg^{-1}|$.  If $|H| =k$ and if $H$ is the only
subgroup of $G$ of order $k$, then, since $|gHg^{-1}| = k$, we must have 
$H = gHg^{-1}$.
Since $g$ was arbitrary, this proves that $H$ is normal in $G$.
\qed

\bigskip

\item[{\bf 10.13.}]
Recall that the {\bf center} of a group $G$ is the set 
\[
Z(G) = \{ x \in G : xg = gx \text{ for all $g \in G$ } \}.
\]
\begin{enumerate}
 
 \item[(a)]
Calculate the center of $S_3$.
 
%%  \item
%% Calculate the center of $GL_2 ( {\mathbb R} )$.
 
 \item[(c)]
Show that the center of any group $G$ is a normal subgroup of $G$. 
 
 \item[(d)]
If $G / Z(G)$ is cyclic, show that $G$ is abelian.\footnote{Hint: Let $Z := Z(G)$.  If
  $G/Z$ is cyclic then there exists $x\in G$ such that for each $a\in G$ 
  there exists $m\in \N$ such that $aZ = x^mZ$.
  Fix $a, b\in G$ and show $ab = ba$ using the fact that $aZ = x^mZ$ and
  $bZ = x^nZ$ for some $m$ and $n$.}
 
\end{enumerate}

\newpage

\noindent {\bf Solution:} 
\begin{enumerate}
 \item[(a)]
Recall the elements of $S_3$ are 
$\{(~), (12), (13), (23), (123), (132)\}$.
The center of a group consists of those elements that commute with
every other element in the group. In case $G = S_3$, the center is trivial.
That is, the only element of $S_3$ that commutes with every other element is the
identity permutation $(~)$.  This can be proved by direct calculation. For each
element $x \in S_3$, there is at least one $y\in S_3$ such that $xy \neq yx$: 
\begin{align*}
  (12)(13) = (132) % \\ & 
  &\neq (123) = (13)(12)\\[4pt]
  (23)(123) = (13) % \\ & 
  &\neq (12) = (123)(23)\\[4pt]
  (132)(13) = (12) % \\ & 
  &\neq (23) = (13)(132)
\end{align*}
With the exception of the identity $(~)$, every element of $S_3$ appears at least
one of the noncommuting products above.  Therefore, $Z(S_3) = \{(~)\}$.

\medskip
 
 \item[(c)]
\emph{Claim:} The center of any group is normal: $Z(G) \triangleleft G$.

\smallskip

\noindent
\emph{Proof 1:}
We use the standard method mentioned above in Problem 10.5.  That is, we fix
arbitrary $z \in Z(G)$ and $a \in G$, and show that $aza^{-1} \in Z(G)$.
Indeed, since $z$ belongs to  $Z(G)$, it commutes with every element of $G$.
Therefore, $aza^{-1} = aa^{-1}z = e z  = z\in Z(G)$.
\qed

\medskip
Although the proof above is nice and short, there is another approach to this
problem that is worth considering because it involves an automorphism,
called an inner automorphism, that will come up again and again. An 
\emph{inner automorphism} is an automorphism $\varphi_a: G \rightarrow G$ of the
form $\varphi_a(x) = axa^{-1}$.  The alternative proof  is also worth studying
because it employs an important idea that we will use again in Problem 6 below
when we prove \emph{Cayley's Representation Theorem}, which says that
every group $G$ is isomorphic to a subgroup of the permutation group $\Sym(G)$.

\medskip

\noindent
\emph{Proof 2:}
For $a \in G$, consider the function 
$\varphi_a : G \rightarrow G$ defined by
$\varphi_a(g) =aga^{-1}$.
It is not hard to show that the function
$\varphi : G \rightarrow \Sym(G)$ defined by $\varphi(a) = \varphi_a$---that is,
the function sending each $a\in G$ to the permutation $\varphi_a\in
\Sym(G)$---is a group homomorphism.   
Moreover, the kernel of $\varphi$ is
\begin{align*}
\ker \varphi &= \{(a, b) \in G^2 : \varphi(a) = \varphi(b)\}
= \{(a, b) \in G^2 : \varphi_a = \varphi_b\}\\
&= \{(a, b) \in G^2 : \varphi_a(g)= \varphi_b(g) \text{ for all } g \in G\}\\
&= \{(a, b) \in G^2 : aga^{-1}= bgb^{-1} \text{ for all } g \in G\}.
\end{align*}
The equivalence class of $\ker \varphi$ that contains the identity element of
$G$ (what the book calls the ``kernel'' of $\varphi$) is the set
\begin{align*}
N_\varphi % &:= \varphi^{-1}(\{e\})\\
&= \{ a \in G : \varphi_a = \varphi_e\}\\
&= \{ a \in G : aga^{-1} = g \text{ for all } g\in G\}\\
&= \{ a \in G : ag = ga \text{ for all } g\in G\}\\
&= Z(G).
\end{align*}
Finally, recall the following {\bf important fact:}\footnote{We proved this in class. It is also proved in the book.} 
\medskip
\begin{quote}
  \emph{For any group homomorphism $\varphi: G \rightarrow H$, the subset
    \[
    N_\varphi = \{g \in G : \varphi(g) = e_H\}
    \]
    of elements mapped by $\varphi$ to the identity of $H$ is a normal subgroup of $G$.}
\end{quote}
\medskip
By taking $\varphi$ to be the conjugation homomorphism as above, we have
$Z(G) = N_\varphi$.  Therefore, $Z(G) \triangleleft G$.
\qed

\medskip

\emph{Remark:} At this point, Proof 2 might seem harder and more complicated
than Proof~1.  However, once you become more comfortable with such arguments
you may find that the easiest and quickest way to see that a certain subgroup is
normal is to simply note that it is (the identity class of) the kernel of a homomorphism.

\bigskip

 \item[(d)]
\emph{Claim:} 
If $G / Z(G)$ is cyclic, then $G$ is abelian.

\smallskip

\noindent
\begin{proof}
For ease of notation, let $Z := Z(G)$.  Assume 
$G/Z$ is cyclic. Then there exists $x\in G$ such that for each $g\in G$ 
  we have $gZ = x^mZ$, for some $m\in \N$.
  Fix $a, b\in G$.  We will show $ab = ba$.
  Let $m$ and $n$ be such that $aZ = x^mZ$ and $bZ = x^nZ$.
  Then $a = x^mz_1$ for some $z_1 \in Z$ and 
  $b = x^nz_2$ for some $z_2 \in Z$.  Therefore,
  \begin{align*}
    a b &= x^m z_1\, x^n z_2 = x^m x^n z_2 z_1 \qquad \text{ (since $z_1 \in Z$) }\\
    &= x^{m+n} z_2 z_1 = x^{n+m} z_2 z_1\\
    &= x^nx^m z_2 z_1 = x^nz_2 x^m z_1 \qquad \text{ (since $z_2 \in Z$) }\\
    &= b a.
  \end{align*}
\end{proof}
 
\end{enumerate}
 
\end{enumerate}

\bigskip

\noindent {\bf Problem 6.} Let $\bG = \<G, \cdot, ^{-1}, e\>$ be a finite group of order $n$.  
Take the set $G$ (the elements of $\bG$) and consider the group of all
permutations of these elements.  This group is sometimes denoted by $\Sym(G)$;
note that it is isomorphic to the symmetric group $S_n$ of permutations of
an $n$-element set.
Now fix an element $a\in G$ and recall that the function
$\lambda_a: G \rightarrow G$, defined by $\lambda_a(g) = a\cdot g$, is a
permutation of the set $G$.  That is, $\lambda_a$ belongs to the
permutation group $\Sym(G)$.


\medskip

\begin{enumerate}[(a)]
\item 
Prove that the function $\lambda: G \rightarrow \Sym(G)$ is a group
homomorphism.  

\medskip

\item What is the kernel of $\lambda$?\footnote{Recall that the kernel of a function $f: X \rightarrow Y$ is the subset of
  $X\times X$ defined by 
\[
\ker f = \{(x_1,x_2) : f(x_1) = f(x_2)\}.
\]
As you have already proved, the kernel is an equivalence relation on $X$.}


\medskip

\item Let $N$ denote the equivalence class of $\ker\lambda$ that contains the
  identity element $e$ of $G$.  Prove that $N$ is a normal subgroup of $G$.
\end{enumerate}

\noindent {\bf Solution:} 

\begin{enumerate}[(a)]
\item 
\emph{Claim:} The function $\lambda: G \rightarrow \Sym(G)$ that takes $a\in G$
to the permutation $\lambda_a\in \Sym(G)$ is a group homomorphism.  
\begin{proof}
The function $\lambda: G \rightarrow \Sym(G)$ is defined for each $a\in G$ by
$\lambda(a) = \lambda_a$.  That is, $\lambda$ takes an element $a\in G$ as input
and outputs the corresponding permutation function $\lambda_a: G \rightarrow G$. 
The latter is defined by $\lambda_a(x) = a\cdot x$.\footnote{We could instead
  represent $\lambda$ as a function of two variables
  $\lambda: G \times G \rightarrow G$, where $\lambda(a,x) = a\cdot x$.
  However, in the context of this problem, and in the second proof of 10.13(d),
  it is cleaner to write $\lambda(a)(x)$, so that  
  $\lambda : G \rightarrow (G \rightarrow G)$. That is, upon input
  $a\in G$ we get a function $\lambda(a): G\rightarrow G$ as output.
  This is sometimes called \href{http://en.wikipedia.org/wiki/Currying}{Currying}. }

We will show that $\lambda$ respects the interpretation of the binary operations
of $G$ and $\Sym(G)$.  That is, we will show that $\lambda(a\cdot b) =
\lambda(a) \circ \lambda(b)$ for all $a, b \in G$.  

Fix $a, b \in G$.  We must show that
the permutation 
$\lambda(a\cdot b) = \lambda_{ab}$
is the same as the permutation 
$\lambda(a) \circ \lambda(b) = \lambda_a \circ \lambda_b$.
Indeed, for all $g \in G$,
\begin{align*}
\lambda_{ab} (g) &= (a\cdot b) \cdot g = a \cdot (b \cdot g) \qquad (\text{associativity})\\
&= a \cdot \lambda_b(g) = \lambda_a (\lambda_b(g)) \\
& = (\lambda_a \circ \lambda_b)(g).
\end{align*}

\end{proof}

\medskip

\item The kernel of $\lambda$ is
  \begin{align*}
\ker \lambda &= \{(a,b) \in G^2 : \lambda(a) = \lambda(b)\}\\
 &= \{(a,b) \in G^2: \lambda_a(x) = \lambda_b(x) \text{ for all $x \in G$}\}\\
 &= \{(a,b) \in G^2: a\cdot x = b\cdot x \text{ for all $x \in G$}\}\\
 &= \{(a,b) \in G^2: a= b\},
  \end{align*}
that is, $\ker \lambda$ is the trivial equivalence relation $0_G := \{(a,a): a\in G\}$.

\bigskip

\item Let $N$ denote the equivalence class of $\ker\lambda$ that contains the
  identity element $e$ of $G$.  Then 
\[
N = \{a \in G : (a, e) \in \ker \lambda\} = \{e\}.
\]
Clearly, $\{e\} \triangleleft G$ since $xex^{-1} = xx^{-1} = e$ for every $x\in G$.
\end{enumerate}

\end{document}
