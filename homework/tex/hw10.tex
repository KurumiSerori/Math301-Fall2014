\documentclass[12pt,reqno]{amsart}
\usepackage[top=2cm, left=2cm,right=2cm,bottom=2cm]{geometry}
\renewcommand{\baselinestretch}{1.2}
\usepackage{amsmath}
\usepackage{amssymb}
\usepackage{scalefnt}
\usepackage{tikz}
\usepackage{color,hyperref,enumerate,multicol}
\definecolor{darkblue}{rgb}{0.0,0.0,0.3}
\hypersetup{colorlinks,breaklinks,
            linkcolor=darkblue,urlcolor=darkblue,
            anchorcolor=darkblue,citecolor=darkblue}
            
\usepackage{algorithm}
\usepackage{algorithmic}
\pagestyle{empty}
\newcommand{\N}{\ensuremath{\mathbb{N}}}
\newcommand{\Z}{\ensuremath{\mathbb{Z}}}
\newcommand{\R}{\ensuremath{\mathbb{R}}}
\newcommand{\bL}{\ensuremath{\mathbf{L}}}
\newcommand{\bP}{\ensuremath{\mathbf{P}}}
\newcommand{\bQ}{\ensuremath{\mathbf{Q}}}
\newcommand{\bA}{\ensuremath{\mathbf{A}}}
\newcommand{\bB}{\ensuremath{\mathbf{B}}}
\newcommand{\bG}{\ensuremath{\mathbf{G}}}
\newcommand{\bH}{\ensuremath{\mathbf{H}}}
\newcommand{\invG}{\ensuremath{\operatorname{inv}^{\bG}}}
\newcommand{\invH}{\ensuremath{\operatorname{inv}^{\bH}}}
\newcommand{\meet}{\ensuremath{\wedge}}
\newcommand{\Meet}{\ensuremath{\bigwedge}}
\newcommand{\<}{\ensuremath{\langle}}
\renewcommand{\>}{\ensuremath{\rangle}}
\newcommand{\join}{\ensuremath{\vee}}
\renewcommand{\emptyset}{\ensuremath{\varnothing}}
\renewcommand{\subset}{\ensuremath{\subsetneq}}
\newcommand{\boldemph}{\emph}
\newcommand{\lcm}{\ensuremath{\operatorname{lcm}}}
\newcommand{\Sym}{\ensuremath{\operatorname{Sym}}}
%\newcommand{\bG}{\ensuremath{\mathbf{G}}}

\newcommand{\probskip}{\vskip1cm}

\begin{document}
\thispagestyle{empty}

\noindent \textbf{Math 301} \hskip5cm {\bf Homework 10} \hfill {\bf Fall 2014}
\vskip1cm
\noindent {\bf Exercises:} Judson 10.1abe, 10.5, 10.10, 10.11, 10.13acd, and
Problem 6 below.\\
{\bf Due date:} Wednesday, 11/05

\bigskip
\begin{enumerate}
%% 1 %%%%%%%%%%%%%%%%%%%%%%%%%%%%%%%%%%%%%%%%%%%%%%%%
\item[{\bf 10.1}] For each of the following groups $G$, determine whether $H$ is a normal
subgroup of $G$. If $H$ is a normal subgroup, write out a Cayley table
for the factor group $G/H$.
\begin{enumerate}
\item[(a)]
$G = S_4$ and $H = A_4$
 \item[(b)]
$G = A_5$ and $H = \{ (1), (123), (132) \}$
%%  \item
%% $G = S_4$ and $H = D_4$
%%  \item
%% $G = Q_8$ and $H = \{ 1, -1, I, -I \}$
\item[(e)]

$G = {\mathbb Z}$ and $H = 5 {\mathbb Z}$
 
\end{enumerate}

\bigskip

\item[{\bf 10.5.}]
Show that the intersection of two normal subgroups is a normal
subgroup. 
 
\bigskip

\item[{\bf 10.10.}]
Let $H$ be a subgroup of index 2 of a group $G$. Prove that $H$ must
be a normal subgroup of $G$. Conclude that $S_n$ is not simple for $n \geq 3$.

\bigskip

\item[{\bf 10.11.}]
If a group $G$ has exactly one subgroup $H$ of order $k$, prove that
$H$ is normal in $G$. 

\bigskip

\item[{\bf 10.13.}]
Recall that the {\bf center} of a group $G$ is the set 
\[
Z(G) = \{ x \in G : xg = gx \text{ for all $g \in G$ } \}.
\]
\begin{enumerate}
 
 \item[(a)]
Calculate the center of $S_3$.
 
%%  \item
%% Calculate the center of $GL_2 ( {\mathbb R} )$.
 
 \item[(c)]
Show that the center of any group $G$ is a normal subgroup of $G$. 
 
 \item[(d)]
If $G / Z(G)$ is cyclic, show that $G$ is abelian.\footnote{Hint: Let $Z := Z(G)$.  If
  $G/Z$ is cyclic then there exists $x\in G$ such that for each $a\in G$ 
  there exists $m\in \N$ such that $aZ = x^mZ$.
  Fix $a, b\in G$ and show $ab = ba$ using the fact that $aZ = x^mZ$ and
  $bZ = x^nZ$ for some $m$ and $n$.}
 
\end{enumerate}
 
\end{enumerate}

\bigskip

\noindent {\bf Problem 6.} Let $\bG = \<G, \cdot, ^{-1}, e\>$ be a finite group of order $n$.  
Take the set $G$ (the elements of $\bG$) and consider the group of all
permutations of these elements.  This group is sometimes denoted by $\Sym(G)$;
note that it is isomorphic to the symmetric group $S_n$ of permutations of
an $n$-element set.
Now fix an element $a\in G$ and recall that the function
$\lambda_a: G \rightarrow G$, defined by $\lambda_a(g) = a\cdot g$, is a
permutation of the set $G$.  That is, $\lambda_a$ belongs to the
permutation group $\Sym(G)$.


\bigskip

\begin{enumerate}[(a)]
\item 
Prove that the function $\lambda: G \rightarrow \Sym(G)$ is a group
homomorphism.  

\medskip

\item What is the kernel of $\lambda$?\footnote{Recall that the kernel of a function $f: X \rightarrow Y$ is the subset of
  $X\times X$ defined by 
\[
\ker f = \{(x_1,x_2) : f(x_1) = f(x_2)\}.
\]
As you have already proved, the kernel is an equivalence relation on $X$.}


\medskip

\item Let $N$ denote the equivalence class of $\ker\lambda$ that contains the
  identity element $e$ of $G$.  Prove that $N$ is a normal subgroup of $G$.
\end{enumerate}


\end{document}
