\documentclass[12pt,reqno]{amsart}
\usepackage[top=2cm, left=2cm,right=2cm,bottom=2cm]{geometry}
\renewcommand{\baselinestretch}{1.2}
\usepackage{amsmath}
\usepackage{amssymb}
\usepackage{scalefnt}
\usepackage{tikz}
\usepackage{color,hyperref,enumerate,multicol}
\definecolor{darkblue}{rgb}{0.0,0.0,0.3}
\hypersetup{colorlinks,breaklinks,
            linkcolor=darkblue,urlcolor=darkblue,
            anchorcolor=darkblue,citecolor=darkblue}
            
\usepackage{algorithm}
\usepackage{algorithmic}
\pagestyle{empty}
\newcommand{\N}{\ensuremath{\mathbb{N}}}
\newcommand{\Z}{\ensuremath{\mathbb{Z}}}
\newcommand{\R}{\ensuremath{\mathbb{R}}}
\newcommand{\bL}{\ensuremath{\mathbf{L}}}
\newcommand{\bP}{\ensuremath{\mathbf{P}}}
\newcommand{\bQ}{\ensuremath{\mathbf{Q}}}
\newcommand{\bA}{\ensuremath{\mathbf{A}}}
\newcommand{\bB}{\ensuremath{\mathbf{B}}}
\newcommand{\bG}{\ensuremath{\mathbf{G}}}
\newcommand{\bH}{\ensuremath{\mathbf{H}}}
\newcommand{\invG}{\ensuremath{\operatorname{inv}^{\bG}}}
\newcommand{\invH}{\ensuremath{\operatorname{inv}^{\bH}}}
\newcommand{\meet}{\ensuremath{\wedge}}
\newcommand{\Meet}{\ensuremath{\bigwedge}}
\newcommand{\<}{\ensuremath{\langle}}
\renewcommand{\>}{\ensuremath{\rangle}}
\newcommand{\join}{\ensuremath{\vee}}
\renewcommand{\emptyset}{\ensuremath{\varnothing}}
\renewcommand{\subset}{\ensuremath{\subsetneq}}
\newcommand{\boldemph}{\emph}
\newcommand{\lcm}{\ensuremath{\operatorname{lcm}}}
\newcommand{\Sym}{\ensuremath{\operatorname{Sym}}}
%\newcommand{\bG}{\ensuremath{\mathbf{G}}}

\newcommand{\probskip}{\vskip1cm}

\begin{document}
\thispagestyle{empty}

\noindent \textbf{Math 301} \hskip3cm {\bf Homework 11 -- Solutions} \hfill {\bf Fall 2014}
\vskip1cm
\noindent {\bf Exercises:} Judson 11.7, 11.11, 11.17, 11.18, 11.19\\
{\bf Due date:} Friday, 11/21

\bigskip
\begin{enumerate}
%% 1 %%%%%%%%%%%%%%%%%%%%%%%%%%%%%%%%%%%%%%%%%%%%%%%%
\item[{\bf 11.7}] 
In the group ${\mathbb Z}_{24}$, let $H = \langle 4 \rangle$ and $N =
\langle 6 \rangle$. 
\begin{enumerate}
 
 \item
List the elements in $HN$ (we usually write $H + N$ for these additive
groups) and $H \cap N$. 
 
 \item
List the cosets in $HN/N$, showing the elements in each coset.
 
 \item
List the cosets in $H/(H \cap N)$, showing the elements in each coset. 
 
 \item
Give the correspondence between $HN/N$ and $H/(H \cap N)$ described in
the proof of the Second Isomorphism Theorem. 
\end{enumerate}

\bigskip
\noindent {\bf Solution:}
First note that, as subgroups of $\Z_{24}$,
\[
H = \< 4 \> = \{0, 4, 8, 12, 16, 20\} \quad \text{ and } \quad 
N = \< 6 \> = \{0, 6, 12, 18\}.
\]
\begin{enumerate}
 \item The elements in $H+N$ are $0, 2, 4, \cdots, 22$.
The elements of $H \cap N$ are $0$ and $12$.
%% \[
%% H+N = \{0, 2, 4, 6, \dots, 22\} \quad \text{ and } \quad
%% H \cap N = \{0, 12\}.
%% \]
 
\medskip 

 \item
The cosets in $(H+N)/N$ are 
%% \begin{align*}
%% N &= \{0, 6, 12, 18\},\\  
%% 2+N &= \{2, 8, 14, 20\},\\
%% 4+N &= \{4, 10, 16, 22\}.
%% \end{align*}
\[
N = \{0, 6, 12, 18\},\qquad
2+N = \{2, 8, 14, 20\},\qquad
4+N = \{4, 10, 16, 22\}.
\]

\medskip 

 \item
The cosets in $H/H \cap N$ are $h + H\cap N$ for each $h\in H$.  That is,
%% \begin{align*}
%% 0 + H\cap N &= \{0, 12\},\\  
%% 4 + H\cap N &= \{4, 16\},\\
%% 8 + H\cap N &= \{8, 20\}.
%% \end{align*}
\[
0 + H\cap N = \{0, 12\},\qquad  
4 + H\cap N = \{4, 16\},\qquad
8 + H\cap N = \{8, 20\}.
\]
(Note that $12 + H \cap N$, $16 + H \cap N$, and $20 + H \cap N$ already appear
in the list.)
\medskip 

 \item
The proof of the Second Isomorphism Theorem begins with a map that takes each 
$h\in H$ to $h+N \in H+N/N$; that is,%
\begin{multicols}{2}
\begin{align*}
0 &\mapsto 0 + N,\\  
4 &\mapsto 4 + N,\\  
8 &\mapsto 8 + N,
\end{align*}

\begin{align*}
12 &\mapsto 12 + N,\\  
16 &\mapsto 16 + N,\\  
20 &\mapsto 20 + N.
\end{align*}
\end{multicols}
\noindent Then, since the kernel subgroup of this map is 
$H\cap N$, the one-to-one correspondence between $H/(H \cap N)$ and
$H+N/N$ is given (by the First Isomorphism Theorem) as follows:
\begin{align*}
0+H\cap N &\longleftrightarrow 0 + N,\\  
4+H\cap N &\longleftrightarrow 4 + N,\\  
8+H\cap N &\longleftrightarrow 2 + N.
\end{align*}

\end{enumerate}

\newpage

\item[{\bf 11.11}]
Show that a homomorphism defined on a cyclic group is completely
determined by its action on the generator of the group.

\medskip

\noindent {\bf Solution:}
Let $G = \<a\>$ be a cyclic group. Let $\varphi$ be a homomorphism from $G$ to
some other group.  We want to show that, for any $x\in G$, we can write the
image $\varphi(x)$ in terms of $\varphi(a)$.  (That's what it means for $\varphi$
to be ``determined by its action on the generator.'')
Indeed, since $x = a^k$ for some $k$, and since $\varphi$ is a homomorphism, we
have $\varphi(x) = \varphi(a^k) = (\varphi(a))^k$.

\bigskip

\item[{\bf 11.17}]
If $H$ and $K$ are normal subgroups of $G$ and $H \cap K = \{ e \}$,
prove that $G$ is isomorphic to a subgroup of $G/H \times G/K$.

\medskip

\newcommand\GHGK{\ensuremath{G/H \times G/K}}

\noindent {\bf Solution:}
Define $\varphi: G \rightarrow G/H \times G/K$ by $\varphi(g) = (gH, gK)$.
First we show $\varphi$ is a homomorphism. By the definition of coset
multiplication and the definition of multiplication in Cartesian
products,
\begin{align*}
\varphi(g_1 g_2) &= 
(g_1 g_2H,g_1 g_2K) = (g_1 Hg_2H,g_1 Kg_2K)\\
 &= (g_1 H,g_1K) (g_2H g_2K) = \varphi(g_1)\varphi(g_2).
\end{align*}
for all $g_1, g_2 \in G$, which proves that $\varphi$ is a homomorphism
from $G$ to $\GHGK$.

Therefore, by the First Isomorphism Theorem, $G/N_\varphi \cong \varphi(G)$, where
$N_\varphi$ is the kernel subgroup associated with $\varphi$.
Moreover, 
the image $\varphi(G) = \{ \varphi(g) : g\in G\}$ is a subgroup of $\GHGK$. 
Finally, note that the
identity element of $\GHGK$ is $(H,K)$, so the kernel subgroup is
\begin{align*}
  N_\varphi &= \{g \in G : \varphi(g) = (H,K)\}\\
&= \{g \in G : (gH,gK) = (H,K)\}\\
&= \{g \in G : gH = H \text{ and } gK=K\}\\
&= \{g \in G : g\in H \text{ and } g\in K\}\\
&= H\cap K.
\end{align*}
By assumption $H\cap K= \{e\}$. Therefore, 
$G = G/\{e\} = G/N_\varphi \cong \varphi(G)$, which is a subgroup of $\GHGK$.\footnote{The equality 
$G = G/\{e\}$ is technically an isomorphism $G \cong G/\{e\}$, 
  since $G/\{e\}$ is a collection of cosets, namely 
  $G/\{e\} = \{g\{e\} : g\in G\}$. However, since $g\{e\} = \{g\}$, it's
  common practice to identify the elements of $G/\{e\} = \{\{g\} : g\in G\}$ with the
  elements of $G$, and say that the quotient group $G/\{e\}$ \emph{is} the group
  $G$.}
\newpage
\item[{\bf 11.18}]
Let $\varphi : G_1 \rightarrow G_2$ be a surjective group homomorphism.
Let $H_1$ be a normal subgroup of $G_1$ and suppose that $\varphi(H_1) =
H_2$.  Prove or disprove that $G_1/H_1 \cong G_2/H_2$.
 
\medskip
\noindent {\bf Solution:}
That this statement is false can be seen by considering the First Isomorphism Theorem.
Let $e_1$ and $e_2$ be the identity elements of $G_1$ and $G_2$, respectively. 
Since $\varphi$ is surjective, $\varphi(G_1) = G_2$ so, 
by the First Isomorphism Theorem,
$G_1/N_\varphi \cong \varphi(G_1) = G_2$, 
where $N_\varphi = \varphi^{-1}(\{e_2\})$ is the kernel subgroup.

Let $H_1 = \{e_1\}$, and suppose $N_\varphi$ strictly contains $H_1$, so
$G_1 \cong G_1/H_1 \ncong G_1/N_\varphi$.  Since $\varphi$ is a homomorphism, we
have $\varphi(H_1) = \varphi(\{e_1\}) = \{e_2\} = H_2$, so
\[
G_1 \cong G_1/H_1 \ncong G_1/N_\varphi \cong G_2 \cong G_2/H_2.
\]

Alternatively, we could show that the statement is false by constructing a
concrete counterexample, such as the following: 
Let $G_1 := \Z_9$ and $G_2 := \Z_9/\<3\>$ and
let $\varphi : \Z_9 \rightarrow \Z_9/\<3\>$ be defined by 
$\varphi(x) = x+\<3\>$.  If $H_1:=\{0\}$, then
$\varphi(\{0\}) = \{0\} = H_2$, and 
\[
G_1/H_1 = \Z_9/\{0\} \ncong \Z_9/\<3\> \cong G_2/\{0\} = G_2/H_2.
\]

\bigskip

\item[{\bf 11.19}]
Let $\phi : G \rightarrow H$ be a group homomorphism.  Show that
$\phi$ is one-to-one if and only if $\phi^{-1}(e) = \{ e \}$.

\medskip

\noindent {\bf Solution:}
($\Rightarrow$) Suppose $\varphi$ is one-to-one.  Since $\varphi$ is a
homomorphism, $\varphi(e_G) = e_H$.  Therefore,
$\varphi(x) = e_H = \varphi(e_G)$ implies $x = e_G$, since $\varphi$ is
one-to-one.  That is, $\varphi^{-1}(\{e_H\})  = \{e_G\}$.

\medskip

\noindent ($\Leftarrow$) Suppose
$\varphi^{-1}(\{e_H\})  = \{e_G\}$, and suppose $x, y\in G$.
We prove that $\varphi(x) = \varphi(y)$ implies $x=y$. 
Indeed, if $\varphi(x) = \varphi(y)$, then
\[
e_H =\varphi(e_G) = \varphi(x^{-1}x) = \varphi(x^{-1})\varphi(x) =
\varphi(x^{-1})\varphi(y) =
\varphi(x^{-1}y).
\]
Therefore, $x^{-1}y$ belongs to the set $\varphi^{-1}(\{e_H\}) = \{e_G\}$, so
$x^{-1}y = e_G$.
Equivalently, $x = y$, so $\varphi$ is one-to-one.

\end{enumerate}


\end{document}


