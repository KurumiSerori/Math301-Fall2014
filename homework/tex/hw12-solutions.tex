\documentclass[12pt,reqno]{amsart}
\usepackage[top=2cm, left=2cm,right=2cm,bottom=2cm]{geometry}
\renewcommand{\baselinestretch}{1.2}
\usepackage{amsmath}
\usepackage{amssymb}
\usepackage{scalefnt}
\usepackage{tikz}
\usepackage{color,hyperref,enumerate,multicol}
\definecolor{darkblue}{rgb}{0.0,0.0,0.3}
\hypersetup{colorlinks,breaklinks,
            linkcolor=darkblue,urlcolor=darkblue,
            anchorcolor=darkblue,citecolor=darkblue}
            
\usepackage{algorithm}
\usepackage{algorithmic}
\pagestyle{empty}
\newcommand{\N}{\ensuremath{\mathbb{N}}}
\newcommand{\Z}{\ensuremath{\mathbb{Z}}}
\newcommand{\R}{\ensuremath{\mathbb{R}}}
\newcommand{\bL}{\ensuremath{\mathbf{L}}}
\newcommand{\bP}{\ensuremath{\mathbf{P}}}
\newcommand{\bQ}{\ensuremath{\mathbf{Q}}}
\newcommand{\bA}{\ensuremath{\mathbf{A}}}
\newcommand{\bB}{\ensuremath{\mathbf{B}}}
\newcommand{\bG}{\ensuremath{\mathbf{G}}}
\newcommand{\bH}{\ensuremath{\mathbf{H}}}
\newcommand{\id}{\ensuremath{\operatorname{id}}}
\newcommand{\invG}{\ensuremath{\operatorname{inv}^{\bG}}}
\newcommand{\invH}{\ensuremath{\operatorname{inv}^{\bH}}}
\newcommand{\meet}{\ensuremath{\wedge}}
\newcommand{\Meet}{\ensuremath{\bigwedge}}
\newcommand{\<}{\ensuremath{\langle}}
\renewcommand{\>}{\ensuremath{\rangle}}
\newcommand{\join}{\ensuremath{\vee}}
\renewcommand{\emptyset}{\ensuremath{\varnothing}}
\renewcommand{\subset}{\ensuremath{\subsetneq}}
\newcommand{\boldemph}{\emph}
\newcommand{\lcm}{\ensuremath{\operatorname{lcm}}}
\newcommand{\Sym}{\ensuremath{\operatorname{Sym}}}
%\newcommand{\bG}{\ensuremath{\mathbf{G}}}
\newcommand{\subnormal}{\ensuremath{\triangleleft}}
\newcommand{\supnormal}{\ensuremath{\triangleright}}
\newcommand{\notsubnormal}{\ensuremath{\ntrianglelefteqslant}}

\newcommand{\probskip}{\vskip1cm}

\begin{document}
\thispagestyle{empty}

\noindent \textbf{Math 301} \hskip5cm {\bf Homework 12} \hfill {\bf Fall 2014}
\vskip1cm
\noindent {\bf Exercises:} 1--7 of the file
\href{https://github.com/williamdemeo/Math301-Fall2014/blob/master/misc/GeneralAlgebraNotes.pdf?raw=true}{GeneralAlgebraNotes.pdf}. {\bf
  Submit:}  Solutions to Exercises 5 and 7.
\\
{\bf Due:} Friday, 12/5.

\medskip

\begin{enumerate}
  %% 1 %%%%%%%%%%%%%%%%%%%%%%%%%%%%%%%%%%%%%%%%%%%%%%%%
\item[{\bf Ex.~5.}] 
    Suppose $K$ and $L$ are normal subgroups of $G$. Prove that $G/K\cap L$ is
    isomorphic to a subgroup of $G/K \times G/L$ (the external direct
    product),\footnote{Note 
      that the expression $G/K\cap L$ can only be interpreted as $G/(K\cap L)$,
      since $(G/K)\cap L$ doesn't make sense.} and compute the index
    of this subgroup in $G/K \times G/L$, 
    in terms of $[G:K]$, $[G:L]$, and $[G:KL]$.

\medskip
%% Below are two alternative solutions to the first part of the problem.  The only
%% difference is that the first makes use of the First Isomorphism Theorem, while
%% the second proves the result directly.  (I recommend students learn the
%% first solution in order to become more comfortable with the First
%% Isomorphism Theorem.) 
\newcommand\GKGL{\ensuremath{G/K \times G/L}}

%% \noindent {\bf Solution 1:}
\noindent {\bf Solution:}
Define $\varphi: G \rightarrow G/K \times G/L$ by $\varphi(g) = (gK, gL)$.
First we show $\varphi$ is a homomorphism. By the definition of coset
multiplication and the definition of multiplication in Cartesian
products,
\begin{align*}
\varphi(g_1 g_2) &= 
(g_1 g_2K,g_1 g_2L) = (g_1 Kg_2K,g_1 Lg_2L)\\
 &= (g_1 K,g_1L) (g_2K g_2L) = \varphi(g_1)\varphi(g_2).
\end{align*}
for all $g_1, g_2 \in G$, which proves that $\varphi$ is a homomorphism
from $G$ to $\GKGL$.
Therefore, by the First Isomorphism Theorem, 
\[
G/N_\varphi \cong \varphi(G),
\] 
where $N_\varphi$ is the kernel subgroup associated with $\varphi$, and 
where $\varphi(G) = \{ \varphi(g) : g\in G\}$ is the image of $G$ under the map
$\varphi$. Since $\varphi$ is a homomorphism, and since homomorphisms map
subgroups to subgroups, it follows that 
$\varphi(G)$ is a subgroup of $\GKGL$. 
Thus $G/N_\varphi$ is isomorphic to a subgroup of $\GKGL$. 

Finally, the identity element of $\GKGL$ is $(K,L)$, so the kernel
subgroup is 
\begin{align*}
  N_\varphi &= \{g \in G : \varphi(g) = (K,L)\}= \{g \in G : (gK,gL) = (K,L)\}\\
&= \{g \in G : gK = K \text{ and } gL=L\}= \{g \in G : g\in K \text{ and } g\in L\}\\
&= K\cap L.
\end{align*}
By assumption $K\cap L= \{e\}$. Therefore, 
$G = G/\{e\} = G/N_\varphi \cong \varphi(G)$, which is a subgroup of $\GKGL$.\footnote{The equality 
$G = G/\{e\}$ is technically an isomorphism $G \cong G/\{e\}$, 
  since $G/\{e\}$ is a collection of cosets, namely 
  $G/\{e\} = \{g\{e\} : g\in G\}$. However, since $g\{e\} = \{g\}$, it's
  common practice to identify the elements of $G/\{e\} = \{\{g\} : g\in G\}$ with the
  elements of $G$, and say that the quotient group $G/\{e\}$ \emph{is} the group
  $G$.}

%% \noindent {\bf Solution 2:}
%%       Define the natural mapping
%%         \[
%%         G/K\cap L \ni g(K\cap L) \mapsto (gK, gL) \in G/K \times G/L.
%%         \]
%%         We must check that this mapping is well-defined and one-to-one, as follows:
%%         \begin{align*}
%%           x(K\cap L) =y(K\cap L)  \quad&  \Leftrightarrow \quad x^{-1}y \in K\cap L\\
%%           & \Leftrightarrow \quad  x^{-1}y \in K \; \text{ and } \; x^{-1}y \in L\\
%%           & \Leftrightarrow \quad xK = yK \;\text{ and } \; xL = yL\\
%%           & \Leftrightarrow \quad (xK, xL) = (yK, yL).
%%         \end{align*}
%%         To be clear, the forward implications $(\Rightarrow)$ prove well-definedness,
%%         the reverse implications $(\Leftarrow)$ one-to-oneness. Also, it is important to
%%         note that these calculations take $G/K$, $G/L$, and $G/K\cap L$ to be \emph{groups},
%%         with multiplications given by $xK yK = xyK$, etc.  This requires that the
%%         subgroups $K$ and $L$, and hence $K\cap L$, be normal.  If, for example, $K$
%%         were not normal, then $G/K$ would not be a group.
   
%%         It remains to check that the mapping is a homomorphism.  Let $\varphi$ denote
%%         the mapping, and  fix two cosets $x(K\cap L)$ and $y(K\cap L)$ in $G/(K\cap L)$.  Then
%%         \begin{align*}
%%         \varphi(x(K\cap L) y(K\cap L)) &= \varphi(xy(K\cap L)) = (xyK,
%%         xyL)=(xKyK,xLyL)\\[4pt]
%%         &=(xK,xL)\,(yK,yL) = \varphi(x(K\cap L))\,\varphi(y(K\cap L)).
%%         \end{align*}
%%         By exhibiting a monomorphism $\varphi: G/K\cap L \hookrightarrow
%%         G/K \times G/L$, we have proved that $G/K\cap L$ can be embedded as a
%%         subgroup in $G/K \times G/L$.

        To find the index of $\varphi(G)$ in $G/K \times G/L$, in terms of $[G:K], [G:L]$ and
        $[G:K\cap L]$, note that 
        $G/K\cap L \cong \varphi(G)$ implies 
        $|\varphi(G)| = |G/K\cap L| = [G: K \cap L]$.
        Also, 
        \[
        |G/K \times G/L| = |G/K| |G/L| = [G:K][G:L].
        \] 
        Therefore, the index of $\varphi(G)$ in $G/K \times G/L$ is
        \[
          [(G/K \times G/L):\varphi(G)] 
          = \frac{|G/K \times G/L|}{|\varphi(G)|}
          = \frac{[G:K][G:L]}{[G: K \cap L]}.
        \]

\bigskip
\noindent {\it Note to students:} The above solution illustrates a {\bf standard method}
of applying the First Isomorphism Theorem to prove that a factor group $G/N$ is
isomorphic to a subgroup of~$H$.  Let us pause to abstract the essence
of this method, so you will know exactly how to apply it in future
situations.  

To begin with, we are presented with the following data: $G$, $N$, and $H$, are
groups, and $N$ is a normal subgroup of $G$ (so $G/N$ is a factor group). Our
objective is to show that $G/N$ is isomorphic to some subgroup of $H$.  (The
specific subgroup $K\leq H$ may or may not be specified in advance.)

\noindent {\bf Standard method:} To show $G/N \cong K$ for some subgroup $K \leq H$, 
\begin{enumerate}
\item Find a function $\varphi: G \rightarrow H$ with the following properties
  (which you must check):
  \begin{enumerate}
  \item $\varphi$ is a homomorphism from $G$ to $H$
  \item the image of $\varphi$ is $K = \varphi(G)$
  \item the kernel subgroup of $\varphi$ is $N = \{g\in G: \varphi(g) = e_H\}$
  \end{enumerate}
\item Conclude by citing the First Isomorphism Theorem, which implies
  \[
  G/N \cong \varphi(G) = K\leq H.
  \]
\end{enumerate}
Notice that even though our initial goal is to find an isomorphism $G/N \cong K$,
we don't start by searching around for a one-to-one and onto homomorphisms
defined on $G/N$.
Rather, we start with the (seemingly) more modest (but equivalent) goal of
finding a homomorphism from $G$ to $H$ with kernel subgroup $N$ and image
$K$. Once we accomplish this, the First Isomorphism Theorem does the rest of the
work for us.

\bigskip

\item[{\bf Ex.~7.}] 
    Let $G$ be a nonabelian simple group.  Let $S_n$ be the symmetric group of all
  permutations on an $n$-element set, and let $A_n$ be the alternating group.
    \begin{enumerate}[{\bf a.}]
    \item 
      Show that if $G$ is a subgroup of $S_n$, $n$ finite, then $G$ is a subgroup
      of $A_n$.
    \item Let $H$ be a proper subgroup of $G$, and, for $g\in G$, let $\lambda_g$ be
      the map of the set of left cosets of $H$ onto themselves defined by
      $\lambda_g(xH) = gxH$.  
      Show that the map $g\mapsto \lambda_g$ is a monomorphism (injective
      homomorphism) of $G$ into the group of permutations of the set of left
      cosets of $H$. 
    \item Let $H$ be a subgroup of $G$ of finite index $n$  and assume $n>1$ (so $H \neq G$).  
      Show that $G$ can be embedded in $A_n$. (That is, show that $G$ is
      isomorphic to a subgroup of $A_n$.
    \item If $G$ is infinite, it has no proper subgroup of finite index.
    \end{enumerate}

\medskip

\noindent {\bf Solution:}
    \begin{enumerate}
    \item Suppose $G$ is a subgroup of $S_n$, $n$ finite, and 
      assume $G$ is nonabelian and simple.  Clearly $G \neq S_n$, since $S_n$ is not simple
      (for example, $A_n \subnormal S_n$).
      Consider $G\cap A_n$. This is a normal subgroup of $G$.  To see this, note
      that if $\sigma \in G\cap A_n$ and $g\in G$, then $g\sigma g^{-1}\in G$ and
      $g\sigma g^{-1}\in A_n$, since $A_n \subnormal S_n$, so 
      $g\sigma g^{-1}\in G \cap A_n$.  
      
      So, $G\cap A_n\subnormal G$ and $G$ is simple, so $G\cap A_n = (e)$ or
      $G\cap A_n = G$.   
      In case $G\cap A_n = G$, we have $G\leq A_n$ and we are done.
      The other case, $G\cap A_n = (e)$, in fact never occurs under the given
      hypotheses. Two alternative proofs of this fact are given below.  

      {\it Proof 1:} Suppose $G\cap A_n = (e)$.  Then $G$ contains only $e$ and odd
      permutations.  
      Let $\zeta\in G$ be an odd permutation.  Then $\zeta^2$ is an even permutation in $G$, so it must
      be $e$.  Thus, every nonidentity element of $G$ has order 2.  Suppose $\eta$ is
      another odd permutation in $G$. Then $\zeta \eta$ is an even permutation in $G$, so
      $\zeta \eta = e$.  Therefore, $\zeta \zeta = e = \zeta \eta$, so $\eta = \zeta$.
      This shows $G$ has only two elements $e$ and $\zeta$.  But then $G\cong \Z_2$ is abelian,
      contradicting our hypothesis. 

      {\it Proof 2:} 
      If $G$ were not a subgroup of $A_n$, then the group
      $G A_n\leq S_n$ would have order larger than $|A_n| = |S_n|/2$. 
      By Lagrange's theorem, then, it would have order
      $|S_n|$.  Therefore, $G A_n = S_n$, and, by the 
      second isomorphism theorem,
      \[
      G/(G\cap A_n) \cong G A_n/A_n = S_n/A_n.
      \]
      This rules out $G\cap A_n = (e)$, since that would give $G \cong S_n/A_n
      \cong \Z_2$ (abelian), contradicting the hypothesis.
      \qed
      ~\\
      {\bf Remark:}  Although the first proof above is completely elementary, the second is worth
      noting since it reveals useful information even when we don't assume $G$ is
      simple and nonabelian.  For example, if $G$ is a subgroup of $S_n$
      containing an odd permutation, then we can show that exactly half of the
      elements of $G$ are even and half are odd.  Indeed, under these conditions 
      we have $G A_n = S_n$ (as in the second proof above), and then the second isomorphism
      theorem implies $|G/(G\cap A_n)| =|S_n/A_n|=2$. Therefore, $|G\cap A_n| =|G|/2$,
      which says that half the elements of $G$ belong to $A_n$, as claimed. 

      \medskip
    \item
      Let $H\leq G$, and let $G/H$ denote the set of left cosets of $G$.  Let
      $\Sym(G/H)$ denote the group of permutations of the set $G/H$.  Then the map 
      $\lambda: G\rightarrow  \Sym(G/H)$---defined by 
      $\lambda(g) = \lambda_g$, where $\lambda_g(xH) = gxH$---is a 
      monomorphism of $G$ into $\Sym(G/H)$.  To prove this, we first show that 
      $\lambda(g) = \lambda_g$ is indeed a bijection of $G/H$---that is,
      $\lambda_g \in \Sym(G/H)$---then we show that 
      $\lambda$ is a homomorphism of $G$, and finally we show that the kernel
      subgroup of $\lambda$ is $\{e\}$ (so $\lambda$ is a monomorphism).

      To see that $\lambda_g$ is one-to-one, observe that
      \begin{align*}
      \lambda_g(xH) = \lambda_g(yH) 
      \quad &\Longrightarrow \quad 
      gxH = gyH
      \quad \Longrightarrow \quad 
      (gy)^{-1}gx \in H\\
      \quad &\Longrightarrow \quad 
      y^{-1}x \in H
      \quad \Longrightarrow \quad 
      xH = yH.
      \end{align*}


      To see that $\lambda_g$ is onto, fix $xH\in G/H$, and note that
      $\lambda_g(g^{-1}xH) = gg^{-1}xH = xH$.

      To see that $\lambda$ is a homomorphism, note that for any $xH \in G/H$ we have
      \[
      \lambda_{g_1 g_2}(xH) = g_1g_2 xH = \lambda_{g_1}(g_2 x H) = 
      \lambda_{g_1}\circ \lambda_{g_2}( x H).
      \]
      That is, $\lambda(g_1 g_2) = \lambda(g_1)\lambda(g_2)$, so $\lambda$
      is a homomorphism.

      Finally, note that the kernel subgroup of $\lambda$, which we denote by
      $N_\lambda$, is a normal subgroup of $G$.  Therefore, as $G$ is simple,
      either $N_\lambda = \{e\}$, or $N_\lambda = G$.  But
      \begin{align*}
      N_\lambda &= \{g\in G: \lambda_g = \id_{G/H}\} \qquad \text{($\id_{G/H} = $ the identity map on $G/H$)}\\
      &= \{g\in G: \lambda_g(xH) = xH \, \text{ for all } x\in G \}\\
      &=\{g\in G: gxH = xH \, \text{ for all } x\in G \}\\
      &=\{g\in G: x^{-1}gxH = H \, \text{ for all } x\in G \}\\
      &=\{g\in G: x^{-1}g x \in H \, \text{ for all } x\in G\}.
      \end{align*}
      If we let $x = e$ in the last expression, we see that $N_\lambda$ must be
      a subgroup of $H$, and since $H$ is properly contained in $G$, we have
      $N_\lambda \neq G$.  Therefore, $N_\lambda = \{e\}$, which proves that
      $\lambda$ is one-to-one, hence a monomorphism.

      \medskip

      \item Let $H$ be a subgroup of $G$ of finite index $n>1$,
        so $H \neq G$.  We must show that $G$ can be embedded into $A_n$.

      In the previous part of the exercise we saw that $\lambda$ embeds $G$ into
      $\Sym(G/H)$, where $G/H$ is the set of left cosets of $H$ in $G$.  
      Now $[G:H] = |G/H| = n$, which implies
      that $\Sym(G/H) \cong S_n$, the group of permutations of $n$ elements.  Therefore,
      $G$ is isomorphic to a subgroup of $S_n$.  By the first part above, then, 
      $G$ is isomorphic to a subgroup of~$A_n$.

      \item Suppose $G$ is infinite.  We must show that $G$ has no proper subgroup of finite
      index.  If, on the contrary, $H\leq G$ with $[G:H] = n$, then, $G$ would be isomorphic
      to a subgroup of the finite group $A_n$, which is clearly impossible,
      since $G$ is infinite.

\end{enumerate}


\end{enumerate}

\end{document}


