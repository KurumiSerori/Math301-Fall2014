\documentclass[12pt,reqno]{amsart}
\usepackage[top=2cm, left=2cm,right=2cm,bottom=2cm]{geometry}
\renewcommand{\baselinestretch}{1.2}
\usepackage{amsmath}
\usepackage{amssymb}
\usepackage{scalefnt}
\usepackage{tikz}
\usepackage{color,hyperref,enumerate,multicol}
\definecolor{darkblue}{rgb}{0.0,0.0,0.3}
\hypersetup{colorlinks,breaklinks,
            linkcolor=darkblue,urlcolor=darkblue,
            anchorcolor=darkblue,citecolor=darkblue}
            
\usepackage{algorithm}
\usepackage{algorithmic}
\pagestyle{empty}
\newcommand{\N}{\ensuremath{\mathbb{N}}}
\newcommand{\Z}{\ensuremath{\mathbb{Z}}}
\newcommand{\R}{\ensuremath{\mathbb{R}}}
\newcommand{\bL}{\ensuremath{\mathbf{L}}}
\newcommand{\bP}{\ensuremath{\mathbf{P}}}
\newcommand{\bQ}{\ensuremath{\mathbf{Q}}}
\newcommand{\bA}{\ensuremath{\mathbf{A}}}
\newcommand{\bB}{\ensuremath{\mathbf{B}}}
\newcommand{\bG}{\ensuremath{\mathbf{G}}}
\newcommand{\bH}{\ensuremath{\mathbf{H}}}
\newcommand{\invG}{\ensuremath{\operatorname{inv}^{\bG}}}
\newcommand{\invH}{\ensuremath{\operatorname{inv}^{\bH}}}
\newcommand{\meet}{\ensuremath{\wedge}}
\newcommand{\Meet}{\ensuremath{\bigwedge}}
\newcommand{\<}{\ensuremath{\langle}}
\renewcommand{\>}{\ensuremath{\rangle}}
\newcommand{\join}{\ensuremath{\vee}}
\renewcommand{\emptyset}{\ensuremath{\varnothing}}
\renewcommand{\subset}{\ensuremath{\subsetneq}}
\newcommand{\boldemph}{\emph}
\newcommand{\lcm}{\ensuremath{\operatorname{lcm}}}
\newcommand{\Sym}{\ensuremath{\operatorname{Sym}}}
%\newcommand{\bG}{\ensuremath{\mathbf{G}}}

\newcommand{\probskip}{\vskip1cm}

\begin{document}
\thispagestyle{empty}

\noindent \textbf{Math 301} \hskip5cm {\bf Homework 13} \hfill {\bf Fall 2014}
\vskip1cm
\noindent {\bf Exercises:} Chapter 14: 1 (except the $GL_2( {\mathbb R} )$
example), 2, 3, 9, 11 (justify!).\\
{\bf Due date:} Friday, 12/12

\medskip
Recall, if $G$ acts on a set $X$ and $x, y \in X$, then $x$ is said
to be \boldemph{$G$-equivalent} to $y$ if there exists a $g \in G$ such that 
$gx =y$. We  write $x \sim_G y$ or $x \sim y$ if $x$ and $y$ are $G$-equivalent. 
In class, we proved that the $G$-equivalent equivalence relation is reflexive and
symmetric. You should check the transitive property on your own to complete the
proof that $\sim$ is an equivalence relation on $X$. 
 
%% \noindent {\bf Exercises:}
\begin{enumerate}
%% 1 %%%%%%%%%%%%%%%%%%%%%%%%%%%%%%%%%%%%%%%%%%%%%%%%
\item[{\bf 14.1}] Each of the examples below describes an action of a group $G$
  on a set $X$, which will give rise to the equivalence relation defined by
  $G$-equivalence.  For each example, compute the equivalence classes of the
  $G$-equivalent equivalence relation.  

\medskip
\noindent {\bf Solution:}

%% {\bf Examples:}
\begin{enumerate}
\item
%\begin{example}{D4_action}
Let $G = D_4$ be the symmetry group of a square, $X = \{ 1, 2, 3, 4 \}$ the
vertices of the square, and suppose $D_4$ consists of the following permutations: 
\[
\{ (1), (13), (24), (1432), (1234), (12)(34), (14)(23), (13)(24) \}.
\]
Then the $D_4$-equivalence classes are the orbits under the action of $D_4$ on
$X$.  For example, the orbit of $1$ is 
\[
\mathcal{O}_1 = 
\{\bar{\sigma} (1) : \sigma \in D_4\} = 
\{\overline{(~)} 1, 
\overline{(1234)} 1,
\overline{(13)} 1, \overline{(1432)} 1\} = \{1, 2, 3, 4\} = X.
\]
So, in this example, we have $X \subseteq \mathcal{O}_1$, and since orbits partition the set $X$
into a union of disjoint orbits (i.e., disjoint $G$-equivalent equivalence
classes), we see that all elements of $X$ are in the same equivalence class.
That is,  $\mathcal{O}_1 = \mathcal{O}_2 = \mathcal{O}_3 = \mathcal{O}_4 = X$.
%% \end{example}
 
 \medskip

\item 
%% \begin{example}{left_action}
Let $X = G$ and let $G$ act on itself by the
left regular action $\lambda_g(x) = gx$. Then the orbit of the identity element
is all of $G$,
\[\mathcal{O}_e = \{\lambda_g(e) : g\in G\} = \{ge : g\in G\} = G.\]
So again we have only a single $G$-equivalent equivalence class,
$\mathcal{O}_g = G$, for all $g\in G$.
%% \end{example}
 
 \medskip

\item
%% \begin{example}{conj_action}
Let $X=G$, a group and let $H\leq G$. Then $G$ is an $H$-set under 
the conjugation action:
$\varphi: H \rightarrow \Sym(G)$ is defined by
$\varphi_h(g) = hgh^{-1}$ for each $h\in H$.  Then the orbits are conjugacy
classes with respect to $H$, that is,
\[
\mathcal{O}_g = \{\varphi_h(g) : h \in H\} = \{hgh^{-1} : h \in H\}.
\]
%% %% \end{example}
 
 \medskip
 
\item
%% \begin{example}{left_coset_action}
Let $H$ be a subgroup of $G$ and let $X = G/H$ be the set of left cosets
of $H$.  The set $G/H$ is a $G$-set under the action
$\lambda: G \rightarrow \Sym(G/H)$ given by 
$\lambda_g(xH) = gxH$.
In this case, all elements of $G/H$ are in the same orbit, so there is just one
$G$-equivalent equivalence class---namely, for every $xH \in G/H$, 
\[
\mathcal{O}_{xH} = \{\lambda_g(xH) : g \in G\} = \{gxH : g \in G\} = 
\{gH : g \in G\} = G/H.
\]
%% \end{example}
\end{enumerate}

\bigskip

\item[{\bf 14.2}] \label{actions}
Compute all $X_g$ and all $G_x$ for each of the following permutation
groups. 
\begin{enumerate}
 
 \item
$X= \{1, 2, 3\}$, \\
$G=S_3=\{(~), (12), (13), (23), (123), (132)  \}$
 
 \item
$X = \{1, 2, 3, 4, 5, 6\}$, \\
$G = \{(~), (12), (345), (354), (12)(345), (12)(354)  \}$
 
\end{enumerate}

\medskip

\noindent {\bf Solution:}

\begin{enumerate}
 
 \item If
$G=\{(~), (12), (13), (23), (123), (132)  \}$, and 
if $X= \{1, 2, 3\}$ is a $G$-set under the left regular action, 
$\bar{g}: x\mapsto gx$, then
 \setlength\multicolsep{0pt}
\begin{multicols}{2}
   \begin{align*}
     X_{(~)} &= \{x\in X : \overline{(~)} x = x\} = X\\
     X_{(123)} &= \{x\in X : \overline{(123)} x = x\} = \emptyset\\
     X_{(132)} &= \{x\in X : \overline{(132)} x = x\} = \emptyset
   \end{align*}

\columnbreak

   \begin{align*}
     X_{(12)} &= \{x\in X : \overline{(12)} x = x\} = \{3\}\\
     X_{(13)} &= \{x\in X : \overline{(13)} x = x\} = \{2\}\\
     X_{(23)} &= \{x\in X : \overline{(23)} x = x\} = \{1\}
   \end{align*}
\end{multicols}

%\vskip-1cm

   \begin{align*}
     G_1 &= \{g\in G : \bar{g}(1) = 1\} = \{(~), (23)\}\\
     G_2 &= \{g\in G : \bar{g}(2) = 2\} = \{(~), (13)\}\\
     G_3 &= \{g\in G : \bar{g}(3) = 3\} = \{(~), (12)\}
   \end{align*}

\bigskip
 
 \item If
$G = \{(~), (12), (345), (354), (12)(345), (12)(354)  \}$, and 
$X = \{1, 2, 3, 4, 5, 6\}$ is a $G$-set under the left regular action, then
\begin{gather*}
  X_{(~)} = X, \quad X_{(12)} =\{3,4,5,6\}, \\
  X_{(345)} =  \{1, 2, 6\} = X_{(345)}, \quad
  X_{(12)(345)} = \{6\} = X_{(12)(354)}.
\end{gather*}
\[
  G_1  = G_2 = \{(~), (345), (354)\}, \quad
  G_3= G_4 = G_5 = \{(~), (12)\} , \quad
  G_6 = G.
\]
\end{enumerate}
 
\bigskip

\item[{\bf 14.3}]
Compute the $G$-equivalent equivalence classes of $X$ for each of the $G$-sets in
Exercise~14.2. For each $x \in X$ verify that 
$|G|=|{\mathcal O}_x| \cdot |G_x|$.  

\medskip

\noindent {\bf Solution:}

\begin{enumerate}
 \item If
$G=\{(~), (12), (13), (23), (123), (132)  \}$ and 
$X= \{1, 2, 3\}$, then under the left regular action,
$     \mathcal{O}_1 =      \mathcal{O}_2 =      \mathcal{O}_3 = \{1,2,3\} = X$,
so, for each $x \in X$, we have $|G| = 6 = 2\cdot 3 = |\mathcal{O}_x|\cdot |G_x|$.

\bigskip
 
 \item If
$G = \{(~), (12), (345), (354), (12)(345), (12)(354)  \}$, and 
$X = \{1, 2, 3, 4, 5, 6\}$, then under the left regular action,
$ \mathcal{O}_1 = \mathcal{O}_2=\{1, 2\}$, and 
$\mathcal{O}_3 = \mathcal{O}_4 = \mathcal{O}_5= \{3,4,5\}$, and 
$\mathcal{O}_6 = \{6\}$.
Therefore, $  |\mathcal{O}_6|\cdot |G_6|  = 6\cdot 1= |G|$, and
\begin{align*}
|\mathcal{O}_x|\cdot |G_x| &= 2\cdot 3  =   |G| \; \text{ for } x \in \{1, 2\}, \text{ and }\\
 |\mathcal{O}_x|\cdot |G_x| &= 3\cdot 2 =   |G|  \; \text{ for } x \in \{3, 4, 5\}.
\end{align*}
\end{enumerate}


\bigskip

\item[{\bf 14.9}]
How many ways can the vertices of an equilateral triangle be colored
using three different colors? 

\medskip

\noindent {\bf Solution:}
Since the problem does not specify whether we should distinguish colorings that
are equivalent up to rotations or reflections, there are three acceptable
answers to this question.  However, the best solution would consider all
reasonable interpretations of the question and provide an answer for each
interpretation. 

\begin{enumerate}
\item \emph{Solution 1:}
If we can distinguish the sides of the triangle, and we don't consider two
colorings to be the same if they differ by reflection or rotation, then there
are $3^3 = \mathbf{27}$ ways to color the triangle with 3 colors.

\item \emph{Solution 2:} If we take colorings that differ by a rotation to
  be equivalent, but we distinguish between colorings that differ by reflection,
  then we consider the rotation subgroup $\{(~), (123), (132)\}$ (which is a
  subgroup of the six-element group of symmetries of the triangle).  Since 
  $(~) = (1)(2)(3)$ has 3 cycles and $(123)$ and $(132)$ each have one cycle, by
  Proposition~14.8 of our textbook, the number of ways to color the sides of a
  triangle with 3 colors up to rotation is
  \[
  \frac{1}{3}(3^3 + 2\cdot 3) = 3^2 + 2 = \mathbf{11}.
  \]

\item \emph{Solution 3:} If we take colorings that differ by a rotation to
  be equivalent, and we also take colorings that differ by a reflection to be
  equivalent, then we factor out by the full group of symmetries of
  the triangle, $G = \{(~), (12), (13), (23), (123), (132)\}$.  Since 
  $(12) = (12)(3)$ has two cycles, as does $(13)$ and $(23)$, then by
  Proposition~14.8, the number of ways to color the sides of a
  triangle with 3 colors up to rotation and reflection is
  \[
  \frac{1}{6}(3^3 + 2\cdot 3 + 3\cdot 3^2) = \frac{1}{6}(2\cdot 3^3 + 2\cdot 3)
  = 3^2 + 1 = \mathbf{10}.
  \]
\end{enumerate}

\bigskip

\item[{\bf 14.11}]
Up to a rotation, how many ways can the faces of a cube be colored
with three different colors?  (Justify any formula you use.)

\medskip

\noindent {\bf Solution:}  There are 24 rotations of the cube.  They are
\begin{itemize}
\item one identity ``rotation'' $(1)(2)(3)(4)(5)(6)$, with 6 cycles in its decomposition,
\item six $90^\circ$ rotations of the form $(1234)(5)(6)$, each with 3 cycles,
\item eight $120^\circ$ rotations of the form $(145)(263)$, each with 2 cycles,
\item six $180^\circ$ rotations of the form $(12)(34)(56)$, each with 3 cycles,
\item three $180^\circ$ rotations of the form $(13)(24)(5)(6)$, each with 4 cycles.
\end{itemize}
Therefore, according to Proposition~14.8, the number of ways to color the faces
of a cube with three different colors is
\[
\frac{1}{24}(3^6 + 12\cdot 3^3 + 8\cdot 3^2 + 3\cdot 3^4) = \mathbf{57}.
\]
 
\end{enumerate}
\end{document}


