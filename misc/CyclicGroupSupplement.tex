
\documentclass[11pt]{article}
\usepackage{graphics}
\usepackage{amsmath}
\usepackage{amssymb}
\usepackage{amsthm}
\usepackage{epsfig}
\usepackage{geometry}
\usepackage{fancyhdr}
\usepackage{graphpap}
\usepackage{pstricks}
\usepackage{pst-node}
\usepackage{scalefnt}
\usepackage{tikz}

\pagestyle{fancy} \lhead{\bf } \chead{\bf Cyclic Group Supplement}
\rhead{} \lfoot{} \cfoot{\thepage} \rfoot{}
\renewcommand{\headrulewidth}{0.6pt}
\renewcommand{\footrulewidth}{0.6pt}
\setlength{\headwidth}{6.5in}
% Fuzz -------------------------------------------------------------------
\hfuzz2pt % Don't bother to report over-full boxes if over-edge is < 2pt
% Line spacing -----------------------------------------------------------
\newlength{\defbaselineskip}
\setlength{\defbaselineskip}{\baselineskip}
\newcommand{\setlinespacing}[1]
           {\setlength{\baselineskip}{#1 \defbaselineskip}}
\newcommand{\doublespacing}{\setlength{\baselineskip}%
                           {2.0 \defbaselineskip}}
\newcommand{\singlespacing}{\setlength{\baselineskip}{\defbaselineskip}}
\setlength{\textwidth}{6.5in} \setlength{\textheight}{9in}
%\setlength{\parindent}{0mm}
\setlength{\oddsidemargin}{.1in}
\setlength{\evensidemargin}{.1in} \setlength{\voffset}{-.5in}
\setlength{\topmargin}{0pt}
% MATH -------------------------------------------------------------------
\newcommand{\A}{{\cal A}}
\newcommand{\h}{{\cal H}}
\newcommand{\s}{{\cal S}}
\newcommand{\W}{{\cal W}}
\newcommand{\D}{\textbf{D}}
\newcommand{\BH}{\mathbf B(\cal H)}
\newcommand{\KH}{\cal  K(\cal H)}
\newcommand{\Real}{\mathbb R}
\newcommand{\R}{\mathbb R}
\newcommand{\C} [1]{{\mathcal #1}}
\newcommand{\B} [1] {{\mathbf #1}}
\newcommand{\Q }{{\mathbb Q}}
\newcommand{\cm} {{\mathbb C}}
\newcommand{\Z} {{\mathbb Z}}
\newcommand{\PP} {{\mathbb P}}
\newcommand{\N }{{\mathbb N}}
\newcommand{\f} {{\mathbb F}}
\newcommand{\gl}{\mathop{\rm GL} }
\newcommand{\ord}{\mathop{\rm ord} }
\newcommand{\lcm}{\mathop{\rm lcm} }
\newcommand{\id}{\mathop{\rm id} }
\newcommand{\GL}{\mathop{\rm GL} }
\newcommand{\Ker}{\mathop{\rm Ker} }
\newcommand{\Adj}{\mathop{\rm Adj} }
\newcommand{\Imm}{\mathop{\rm Im} }
\newcommand{\RR}{\mathbb R}
\newcommand{\NN}{\mathbb N}
\newcommand{\CC}{\mathbb C}
\newcommand{\Complex}{\mathbb C}
\newcommand{\Field}{\mathbb F}
\newcommand{\RPlus}{[0,\infty)}
\newcommand{\norm}[1]{\left\Vert#1\right\Vert}
\newcommand{\essnorm}[1]{\norm{#1}_{\text{\rm\normalshape ess}}}
\newcommand{\abs}[1]{\left\vert#1\right\vert}
\newcommand{\set}[1]{\left\{#1\right\}}
\newcommand{\seq}[1]{\left<#1\right>}
\newcommand{\eps}{\varepsilon}
\newcommand{\To}{\longrightarrow}
\newcommand{\RE}{\operatorname{Re}}
\newcommand{\IM}{\operatorname{Im}}
\newcommand{\Poly}{{\cal{P}}(E)}
\newcommand{\EssD}{{\cal{D}}}
\newcommand{\order}[1]{\ensuremath{|#1|}}
\newcommand{\units}[1]{\ensuremath{U(#1)}}
\newcommand{\divides}{\ensuremath{\mid}}
\newcommand{\<}{\ensuremath{\langle}}
\renewcommand{\>}{\ensuremath{\rangle}}
\newcommand{\eye}{\ensuremath{e}}

% THEOREMS ---------------------------------------------------------------
\theoremstyle{plain}
\newtheorem{thm}{Theorem}%[section]
\newtheorem{cor}[thm]{Corollary}
\newtheorem{lem}[thm]{Lemma}
\newtheorem{prop}[thm]{Proposition}
%
\theoremstyle{definition}
\newtheorem{defn}[thm]{Definition}
%
%\theoremstyle{remark}
\newtheorem{rem}[thm]{Remark}
\newtheorem{rems}[thm]{Remarks}
\newtheorem{ex}[thm]{Example}
\newtheorem{exs}[thm]{Examples}
\begin{document}

\begin{thm}
  Let $g$ be an element of a group $G$ and write
$$\langle g\rangle = \set{g^k: k\in \Z}.$$
Then $\langle g\rangle$ is a subgroup of $G$.
\end{thm}
\begin{proof}  Since $\eye=g^0$, $\eye\in \langle g\rangle$.  Suppose $a$,
$b\in \langle g\rangle$.  Then $a=g^k$, $b=g^m$ and
$ab=g^kg^m=g^{k+m}$. Hence $ab\in \langle g\rangle$ (note that
$k+m\in \Z$).  Moreover, $a^{-1} = (g^k)^{-1} = g^{-k}$ and $-k\in
\Z$, so that $a^{-1}\in \langle g\rangle$.  Thus, we have checked
the three conditions necessary for $\langle g\rangle$ to be a
subgroup of $G$.
\end{proof}
\begin{defn}  If $g\in G$, then the subgroup $\langle g\rangle = \{g^k:
k\in \Z\}$ is called the {\bf cyclic subgroup of $G$ generated by
$g$}, If $G=\langle g\rangle$, then we say that $G$ is a {\bf
cyclic group} and that $g$ is a {\bf generator} of $G$.
\end{defn}
\begin{exs}  \begin{enumerate}
\item If $G$ is any group  then
$\{\eye\} =\langle \eye\rangle$ is a cyclic subgroup of $G$.

\item  The group $G=\{1,\, -1,\, i,\, -i\} \subseteq \cm^*$ (the
group operation  is multiplication of complex numbers) is cyclic
with generator $i$.  In fact $\langle i\rangle = \{i^0=1, i^1=i,
i^2=-1, i^3=-i\} = G$.  Note that $-i$ is also a generator for $G$
since $\langle -i\rangle =\{(-i)^0=1, (-i)^1=-i, (-i)^2=-1,
(-i)^3=i\}=G$.  Thus a cyclic group may have more than one
generator.  However, not all elements of $G$ need be generators.
For example $\langle -1\rangle = \{1, -1\}\ne G$ so $-1$ is not a
generator of $G$.

\item The group $G=\units{7}=$ the group of units in $\Z_7$ is
a cyclic group with generator $3$.  Indeed,
$$\langle 3\rangle = \{ 1=3^0, 3=3^1, 2=3^2, 6=3^3, 4=3^4, 5=3^5\}=G.$$
Note that $5$ is also a generator of $G$, but that
$\langle 2\rangle = \{1, 2, 4\}\ne G$ so that 2 is not a generator of $G$.

\item  $G=\langle \pi\rangle =\{\pi^k:k\in \Z\}$ is a cyclic
subgroup of $\R^*$.

\item  The group $G=\units{8}$ is not cyclic.  Indeed, since
$\units{8}=\{1, \, 3,\, 5,\,7\}$ and $\langle 1\rangle = \{1\}$,
$\langle 3\rangle = \{1, 3\}$, $\langle 5\rangle=\{1, 5\}$,
$\langle 7\rangle =\{1, 7\}$, it follows that $\units{8}\ne \langle
a\rangle$ for any $a\in \units{8}$.
\end{enumerate}
\medskip
If a group $G$ is written additively, then the identity
element is denoted $0$,   the inverse of $a\in G$ is denoted $-a$,
and the powers of $a$ become $na$ in additive notation.  Thus,
with this notation, the cyclic subgroup of $G$ generated by $a$ is
$\langle a \rangle = \{na: n\in \Z\}$, consisting of all the
multiples of $a$. Among groups that are normally written
additively, the following are two examples of cyclic groups.
\begin{enumerate}
\addtocounter{enumi}{5} \item  The integers $\Z$ are a cyclic
group. Indeed, $\Z=\langle 1\rangle$ since each integer $k=k\cdot
1$ is a multiple of $1$, so $k\in \langle 1\rangle$ and $\langle
1\rangle =\Z$.  Also, $\Z=\langle -1\rangle$ because $k=(-k)\cdot
(-1)$ for each $k\in \Z$.

\item  $\Z_n$ is a cyclic group under addition with generator $1$.
\end{enumerate}
\end{exs}
\begin{thm} 
\label{thm:4}
   Let $g$ be an element of a group $\mathbf{G} = \<G, \cdot, ^{-1}, e\>$.  
Then there are two possibilities for the cyclic subgroup $\langle g\rangle$.
\begin{enumerate}
\item[{ }]{\bf Case 1:}  The cyclic subgroup $\langle g\rangle$ is finite.  In this
case, there exists a smallest positive integer $n$ such that
$g^n=\eye$ and we have
\begin{enumerate}
\item $g^k=\eye$ if and only if $n\divides k$.
\item  $g^k=g^m$ if and only if $k\equiv m \pmod{n}$.
\item  $\langle g\rangle =\{\eye,\, g,\, g^2,\, \ldots,\, g^{n-1}\}$ and
the elements $\eye,\, g,\, g^2,\, \ldots,\, g^{n-1}$ are distinct.
\end{enumerate}

\item[{ }]{\bf Case 2:}  The cyclic subgroup $\langle g\rangle $ is infinite.
Then
\begin{enumerate}\addtocounter{enumii}{3}
\item $g^k=\eye$ if and only if $k=0$.
\item  $g^k=g^m$ if and only if $k=m$.
\item  $\langle g\rangle =\{ \ldots,\, g^{-3},\, g^{-2},\, g^{-1},\, \eye,\, g,\,g^2,\, g^3,\, \ldots\}$ and all of
these powers of $g$ are distinct.
\end{enumerate}
\end{enumerate}
\end{thm}
\begin{proof}  {\bf Case 1.} Since $\langle g\rangle $ is finite, the powers
$g$, $g^2$, $g^3$, $\ldots$ are not all distinct, so let $g^k=g^m$
with $k<m$.  Then $g^{m-k} =\eye$ where $m-k>0$.  Hence there is a
positive integer $l$ with $g^l=\eye$.  Hence there is a smallest such
positive integer.  We let $n$ be this smallest positive integer,
i.e., $n$ is the smallest positive integer such that $g^n=\eye$.

(a)  If $n\divides k$ then $k=qn$ for some $q\in n$.  Then
$g^k=g^{qn}=(g^n)^q=\eye^q=\eye$.   Conversely, if $g^k=\eye$, use the
division algorithm to write $k=qn+r$ with $0\le r<n$.  Then
$g^r=g^k(g^n)^{-q} = \eye\eye^{-q} =\eye$.  Since $r<n$, this contradicts
the minimality of $n$ unless $r=0$.  Hence $r=0$ and $k=qn$ so
that $n\divides k$.

(b)  $g^k=g^m$ if and only if $g^{k-m} = \eye$.  Now apply Part (a).

(c)  Clearly, $\{\eye,\, g,\, g^2,\, \ldots,\, g^{n-1}\} \subseteq
\langle g\rangle$.   To prove the other inclusion, let $a\in
\langle g\rangle$.  Then $a=g^k$ for some $k\in \Z$.  As in Part
(a), use the division algorithm to write $k=qn+r$, where $0\le
r\le n-1$.  Then
$$a=g^k=g^{qn+r}=(g^n)^qg^r=\eye^qg^r=g^r\in \{\eye,\, g,\, g^2,\, \ldots,\, g^{n-1}\}$$
which shows  that $\langle g\rangle \subseteq \{\eye,\, g,\, g^2,\,
\ldots,\, g^{n-1}\}$, and hence that
$$\langle g\rangle = \{\eye,\, g,\, g^2,\, \ldots,\, g^{n-1}\}.$$
Finally, suppose that $g^k=g^m$ where $0\le k\le m\le n-1$.   Then
$g^{m-k} =\eye$ and $0\le m-k <n$.  This implies that $m-k=0$ because
$n$ is the smallest positive power of $g$ which equals $\eye$.  Hence
all of the elements $\eye,\, g,\, g^2,\, \ldots,\, g^{n-1}$ are
distinct.

{\bf Case 2.}  (d)  Certainly,  $g^k=\eye$ if $k=0$.  If $g^k=\eye$,
$k\ne 0$, then $g^{-k} = (g^k)^{-1} = \eye^{-1} = \eye$, also.  Hence
$g^n=\eye$ for some $n>0$, which implies that $\langle g\rangle$ is
finite by the proof of Part (c), contrary to our hypothesis in
Case 2.  Thus $g^k=\eye$ implies that $k=0$.

(e)  $g^k=g^m$ if and only if $g^{k-m} =\eye$.  Now apply Part (d).

(f)  $\langle g\rangle =\{g^k: k\in \Z\}$  by definition of
$\langle g\rangle$, so all that remains is to check that these
powers are distinct.  But this is the content of Part (e).
\end{proof}

Recall that if $g$ is an element of a  group $G$, then the {\bf
order} of $g$ is the smallest positive integer $n$ such that
$g^n=\eye$, and it is denoted $\order{g}=n$.  If there is no such positive
integer, then we say that $g$ has {\bf infinite order}, denoted
$\order{g}=\infty$.  By Theorem~\ref{thm:4}, the concept of order of an element
$g$ and order of the cyclic subgroup generated by $g$ are the
same.

\begin{cor}    If $g$ is an element of a group $G$, then
$\order{g}=|\langle g\rangle|$.
\end{cor}
\begin{proof}  This is immediate from Theorem~\ref{thm:4}, Part (c).
\end{proof}

If $G$ is a cyclic group  of order $n$,  then it is easy to
compute the order of all elements of $G$.  This is the content of
the following result.

\begin{thm}
\label{thm:6}
  Let $G=\langle g\rangle $ be a cyclic
group of order $n$, and let $0\le k\le n-1$.  If $m=\gcd (k, n)$,
then $\order{g^k}=\dfrac{n}{m}$.
\end{thm}

\begin{proof}  Let $k=ms$ and $n=mt$.  Then
$(g^k)^{n/m}=g^{kn/m}=g^{msn/m}=(g^n)^s=\eye^s=\eye$.  Hence $n/m$
divides $\order{g^k}$ by Theorem~\ref{thm:4} Part (a).  Now suppose that
$(g^k)^r=\eye$.  Then $g^{kr} = \eye$, so by Theorem~\ref{thm:4} Part (a), $n\divides
kr$. Hence $$\frac{n}{ m}\,\Big|\,\frac{k}{ m} r$$ and since $n/m$
and $k/m$ are relatively prime, it follows that $n/m$ divides $r$.
Hence $n/m$ is the smallest power of $g^k$ which equals $\eye$, so
$\order{g^k}=n/m$.
\end{proof}

\begin{thm}  Let $G=\langle g\rangle$ be a cyclic group where
$\order{g}=n$. Then $G=\langle g^k\rangle $ if and only if $\gcd (k, n)
= 1$.
\end{thm}

\begin{proof}  By Theorem~\ref{thm:6},  if $m=\gcd (k, n)$, then $\order{g^k}=n/m$.  But
$G=\langle g^k\rangle $ if and only if $\order{g^k}=|G|=n$ and this
happens if and only if $m=1$, i.e., if and only if $\gcd(k, n)=1$.
\end{proof}

\begin{ex}
  If $G=\langle g\rangle$ is a cyclic group of order 12, then
the generators of $G$ are the powers $g^k$ where $\gcd (k, 12) =
1$, that is $g$, $g^5$, $g^7$, and $g^{11}$.  In the particular
case of the additive cyclic group $\Z_{12}$, the generators are
the integers 1, 5, 7, 11 $\pmod{12}$.
\end{ex}
Now we ask what the subgroups of a  cyclic group look like.  The
question is completely answered by Theorem~\ref{thm:10}.  
Theorem~\ref{thm:9} is a preliminary, but important, result.

\begin{thm}
\label{thm:9}   Every subgroup of a cyclic group is cyclic.
\end{thm}

\begin{proof}
Suppose that $G=\langle g\rangle =\{g^k: k\in \Z\}$ is a cyclic
group and let $H$ be a subgroup of $G$.  If $H=\{\eye\}$, then $H$ is
cyclic, so we assume that $H\ne \{\eye\}$, and let $g^k\in H$ with
$g^k\ne \eye$.  Then, since $H$ is a subgroup, $g^{-k} =
(g^k)^{-1}\in H$.  Therefore, since $k$ or $-k$ is positive,  $H$
contains a positive power of $g$, not equal to \eye.  So let $m$ be
the smallest positive integer such that $g^m\in H$.  Then,
certainly all powers of $g^m$ are also in $H$, so we have $\langle
g^m\rangle \subseteq H$.  We claim that this inclusion is an
equality.  To see this, let $g^k$ be any element of $H$ (recall
that all elements of $G$, and hence $H$, are powers of $g$ since
$G$ is cyclic).  By the division algorithm, we may write $k=qm+r$
where $0\le r<m$.  But $g^k=g^{qm+r}=g^{qm}g^r=(g^m)^qg^r$ so that
$$g^r=(g^m)^{-q}g^k\in H.$$ Since $m$ is the smallest positive
integer with $g^m\in H$ and $0\le r<m$, it follows that we must
have $r=0$.  Then $g^k=(g^m)^q\in \langle g^m\rangle$. Hence we
have shown that $H\subseteq \langle g^m\rangle$ and hence
$H=\langle g^m\rangle$.  That is $H$ is cyclic with generator
$g^m$ where $m$ is the smallest positive integer for which $g^m
\in H$.
\end{proof}

\begin{thm}[Fundamental Theorem of Finite Cyclic Groups]
\label{thm:10}
Let $G=\langle g\rangle$ be a cyclic group of order $n$.
\begin{enumerate}
\item  If $H$ is any subgroup of $G$, then $H=\langle g^d\rangle$ for some $d\divides n$.
\item  If $H$ is any subgroup of $G$ with $|H|=k$, then $k\divides n$.
\item  If $k\divides n$, then $\langle g^{n/k}\rangle$ is the unique subgroup
of $G$ of order $k$.
\end{enumerate}
\end{thm}

\begin{proof} \begin{enumerate}
\item By Theorem 9, $H$ is a cyclic group and since $|G|=n<\infty$,
it follows that $H=\langle g^m\rangle$ where $m>0$.  Let
$d=\gcd(m, n)$.  Since $d\divides n$ it is sufficient to show that
$H=\langle g^d\rangle$.  But $d\divides m$ also, so $m=qd$.  Then
$g^m=(g^d)^q$ so $g^m\in \langle g^d\rangle$.  Hence $H=\langle
g^m\rangle \subseteq \langle g^d\rangle$.  But $d=rm+ sn$, where
$r$, $s\in \Z$, so
$$g^d=g^{rm+sn}=g^{rm}g^{sn}=(g^m)^r(g^n)^s=(g^m)^r(\eye)^s=
(g^m)^r\in\langle g^m\rangle = H.$$
This shows that $\langle g^d \rangle \subseteq H$ and  hence
$\langle g^d\rangle = H$.

\item By Part (a), $H=\langle g^d\rangle$ where $d\divides n$.  Then $k=|H|=
n/d$ so $k\divides n$.

\item Suppose that $K$ is any subgroup of $G$ of order $k$.  By
Part (a), let $K=\langle g^m\rangle$ where $m\divides n$.  Then
Theorem 6 gives $k=|K| = |g^m| = n/m$.  Hence $m=n/k$, so
$K=\langle g^{n/k}\rangle$.  This proves (c).
\end{enumerate}
\end{proof}

\begin{rem} Part (b) of Theorem~\ref{thm:10} is actually true for {\it any} finite
group $G$, whether or not it is cyclic.  This result is Lagrange's
Theorem (Theorem 6.5, Page 86 of Judson).
\end{rem}


The subgroups of a group $G$ can be diagrammatically  illustrated
by listing the subgroups, and indicating inclusion relations by
means of a line directed upward from $H$ to $K$ if $H$ is a
subgroup of $K$.  Such a scheme is called the {\bf lattice
diagram} for the subgroups of the group $G$.  We will illustrate
by determining the lattice diagram for all the subgroups of a
cyclic group $G=\langle g\rangle$ of order 12.  Since the order of
$g$ is 12, Theorem 10 (c) shows that there is exactly one subgroup
$\langle g^d\rangle$ for each divisor $d$ of 12.  The divisors of
$12$ are 1, 2, 3, 4, 6, 12.  Then the unique subgroup of $G$ of
each of these orders is, respectively,
\[
\{\eye\}=\langle g^{12}\rangle,\quad \langle g^6\rangle, \quad
\langle g^4\rangle,\quad \langle g^3\rangle, \quad \langle
g^2\rangle , \quad \langle g\rangle = G.
\]
Note that $\langle
g^m\rangle \subseteq \langle g^k\rangle$ if and only if $k\divides m$.
Hence the lattice diagram of $G$ is:

\begin{center}
\begin{tikzpicture}[scale=0.75]
  \node (1) at (0,0) {$\langle \eye \rangle$}; 
  \node (6) at (2,2) {$\langle g^6 \rangle$}; 
  \node (4) at (-2,2) {$\langle g^4 \rangle$}; 
  \node (2) at (0,4) {$\langle g^2 \rangle$}; 
  \node (3) at (4,4) {$\langle g^3 \rangle$}; 
  \node (G) at (2,6) {$\langle G \rangle$}; 

  \draw (1) to (6) to (2) to (4) to (1);
  \draw (6) to (3) to (G) to (2) to (4);

\end{tikzpicture}
\end{center}

Finally, here is one more result about cyclic groups that is sometimes
useful (for example, in the proof that $U(4n)$ is cyclic---see Homework 5
solutions).
\begin{lem}
A cyclic group contains at most one element of order 2.  
\end{lem}
\noindent Put another way, an involution\footnote{Recall, an \emph{involution} is an element
  of order 2.} of a cyclic group, if it exists, is unique.
\begin{proof}
Let $G = \<a \>$ be a cyclic group.

If $G$ is infinite, then there are no elements of order 2.
So, assume the order of $G$ is finite: $|G| = n < \infty$.
If $n=1$, then $G = \<e\>$; if $n=2$, then $G = \{e, a\}$ and $a^2 = e$.
In both cases, there is nothing to prove.

Suppose $n>2$, and let $x, y \in G$ be two non-identity elements of $G$, say, 
$x = a^j$ and $y = a^k$, where $1< j, k < n$. If $x^2 = e$, then $a^{2j}=e$.  
Therefore $n$ divides $2j$ (by Theorem 4(a)).
But $j < n$ implies $2j<2n$, so the only way to have $n\divides 2j$ is
$n=2j$. If $y^2 = e$, then the same argument applied to $k$ yields $n=2k$.
It follows that if $x^2 = e = y^2$, then $j=k$ and so $x = a^{j}=a^k = y$. 
Hence involutions of cyclic groups are unique.   
\end{proof}
\end{document}

