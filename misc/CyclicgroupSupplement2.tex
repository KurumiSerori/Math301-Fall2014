\documentclass[12pt,reqno]{amsart}
\usepackage[top=2cm, left=2cm,right=2cm,bottom=2cm]{geometry}
\renewcommand{\baselinestretch}{1.2}
\usepackage{amsmath}
\usepackage{amssymb}
\usepackage{color,hyperref,enumerate,multicol}
\definecolor{darkblue}{rgb}{0.0,0.0,0.3}
\hypersetup{colorlinks,breaklinks,
            linkcolor=darkblue,urlcolor=darkblue,
            anchorcolor=darkblue,citecolor=darkblue}
\pagestyle{empty}
\newcommand{\N}{\ensuremath{\mathbb{N}}}
\newcommand{\Z}{\ensuremath{\mathbb{Z}}}
\newcommand{\R}{\ensuremath{\mathbb{R}}}
\newcommand{\<}{\ensuremath{\langle}}
\renewcommand{\>}{\ensuremath{\rangle}}
\newcommand{\meet}{\ensuremath{\wedge}}
\newcommand{\Meet}{\ensuremath{\bigwedge}}
\newcommand{\join}{\ensuremath{\vee}}
\renewcommand{\emptyset}{\ensuremath{\varnothing}}
\renewcommand{\subset}{\ensuremath{\subsetneq}}
\newcommand{\boldemph}{\emph}
\newcommand{\lcm}{\operatorname{lcm}}

\newtheorem{lemma}{Lemma}

\begin{document}
\thispagestyle{empty}
\begin{center}
{\bf Cyclic Group Supplement 2}
\end{center}
\hrule

\vskip1cm

\begin{lemma}
A cyclic group contains at most one element of order 2.  
\end{lemma}
\noindent Put another way, an involution\footnote{Recall, an \emph{involution} is an element
  of order 2.} of a cyclic group, if it exists, is unique.
\begin{proof}
Let $G = \<a \>$ be a cyclic group.

If $G$ is infinite, then there are no elements of order 2.
So, assume the order of $G$ is finite: $|G| = n < \infty$.
If $n=1$, then $G = \<e\>$; if $n=2$, then $G = \{e, a\}$ and $a^2 = e$.
In both cases, there is nothing to prove.

Suppose $n>2$, and let $x, y \in G$ be two non-identity elements of $G$, say, 
$x = a^j$ and $y = a^k$, where $1< j, k < n$. If $x^2 = e$, then $a^{2j}=e$.  
Therefore $n$ divides $2j$ (by Theorem 4(a) of Cyclic Group Supplement 1).
But $j < n$ implies $2j<2n$, so the only way to have $n|2j$ is
$n=2j$. If $y^2 = e$, then the same argument applied to $k$ yields $n=2k$.
It follows that if $x^2 = e = y^2$, then $j=k$ and so $x = a^{j}=a^k = y$. 
Hence involutions of cyclic groups are unique.   
\end{proof}

\end{document}
