\documentclass[12pt]{article}
\usepackage{graphics}
\usepackage{amsmath}
\usepackage{amssymb}
\usepackage{amsthm}
\usepackage{epsfig}
\usepackage{geometry}
\usepackage{fancyhdr}
\usepackage{graphpap}
\usepackage{pstricks}
\usepackage{pst-node}
\usepackage{scalefnt}
\usepackage{tikz}

\pagestyle{fancy} \lhead{\bf } \chead{\bf On the Generation of Subgroups}
\rhead{} \lfoot{} \cfoot{\thepage} \rfoot{}
\renewcommand{\headrulewidth}{0.6pt}
\renewcommand{\footrulewidth}{0.6pt}
\setlength{\headwidth}{6.5in}
% Fuzz -------------------------------------------------------------------
\hfuzz2pt % Don't bother to report over-full boxes if over-edge is < 2pt
% Line spacing -----------------------------------------------------------
\newlength{\defbaselineskip}
\setlength{\defbaselineskip}{\baselineskip}
\newcommand{\setlinespacing}[1]
           {\setlength{\baselineskip}{#1 \defbaselineskip}}
\newcommand{\doublespacing}{\setlength{\baselineskip}%
                           {2.0 \defbaselineskip}}
\newcommand{\singlespacing}{\setlength{\baselineskip}{\defbaselineskip}}
\setlength{\textwidth}{6.5in} \setlength{\textheight}{9in}
%\setlength{\parindent}{0mm}
\setlength{\oddsidemargin}{.1in}
\setlength{\evensidemargin}{.1in} \setlength{\voffset}{-.5in}
\setlength{\topmargin}{0pt}
% MATH -------------------------------------------------------------------
\newcommand{\A}{{\cal A}}
\newcommand{\h}{{\cal H}}
\newcommand{\s}{{\cal S}}
\newcommand{\W}{{\cal W}}
\newcommand{\D}{\textbf{D}}
\newcommand{\BH}{\mathbf B(\cal H)}
\newcommand{\KH}{\cal  K(\cal H)}
\newcommand{\Real}{\mathbb R}
\newcommand{\R}{\mathbb R}
\newcommand{\C} [1]{{\mathcal #1}}
\newcommand{\B} [1] {{\mathbf #1}}
\newcommand{\Q }{{\mathbb Q}}
\newcommand{\cm} {{\mathbb C}}
\newcommand{\Z} {{\mathbb Z}}
\newcommand{\PP} {{\mathbb P}}
\newcommand{\N }{{\mathbb N}}
\newcommand{\f} {{\mathbb F}}
\newcommand{\gl}{\mathop{\rm GL} }
\newcommand{\ord}{\mathop{\rm ord} }
\newcommand{\lcm}{\mathop{\rm lcm} }
\newcommand{\id}{\mathop{\rm id} }
\newcommand{\GL}{\mathop{\rm GL} }
\newcommand{\Ker}{\mathop{\rm Ker} }
\newcommand{\Adj}{\mathop{\rm Adj} }
\newcommand{\Imm}{\mathop{\rm Im} }
\newcommand{\RR}{\mathbb R}
\newcommand{\NN}{\mathbb N}
\newcommand{\CC}{\mathbb C}
\newcommand{\Complex}{\mathbb C}
\newcommand{\Field}{\mathbb F}
\newcommand{\RPlus}{[0,\infty)}
\newcommand{\norm}[1]{\left\Vert#1\right\Vert}
\newcommand{\essnorm}[1]{\norm{#1}_{\text{\rm\normalshape ess}}}
\newcommand{\abs}[1]{\left\vert#1\right\vert}
\newcommand{\set}[1]{\left\{#1\right\}}
\newcommand{\seq}[1]{\left<#1\right>}
\newcommand{\eps}{\varepsilon}
\newcommand{\To}{\longrightarrow}
\newcommand{\RE}{\operatorname{Re}}
\newcommand{\IM}{\operatorname{Im}}
\newcommand{\Poly}{{\cal{P}}(E)}
\newcommand{\EssD}{{\cal{D}}}
\newcommand{\order}[1]{\ensuremath{|#1|}}
\newcommand{\units}[1]{\ensuremath{U(#1)}}
\newcommand{\divides}{\ensuremath{\mid}}
\newcommand{\<}{\ensuremath{\langle}}
\renewcommand{\>}{\ensuremath{\rangle}}
\newcommand{\eye}{\ensuremath{e}}

% THEOREMS ---------------------------------------------------------------
\theoremstyle{plain}
\newtheorem{thm}{Theorem}%[section]
\newtheorem{cor}[thm]{Corollary}
\newtheorem{lem}[thm]{Lemma}
\newtheorem{prop}[thm]{Proposition}
%
\theoremstyle{definition}
\newtheorem{defn}[thm]{Definition}
%
%\theoremstyle{remark}
\newtheorem{rem}[thm]{Remark}
\newtheorem{rems}[thm]{Remarks}
\newtheorem{ex}[thm]{Example}
\newtheorem{exs}[thm]{Examples}
\begin{document}

Let $G$ be a group and suppose $X$ is a nonempty set of elements of $G$.
The {\bf subgroup generated by $X$} is the smallest subgroup of $G$ that
contains $X$.  
For example, the subgroup of $\Z_{12}$ generated by the set $\{4, 6\}$ is 
$\{0, 2, 4, 6, 8, 10\}$ (explained below).
For a single element $g \in G$, we often denote the subgroup generate
by the set $\{g\}$ by $\<g\>$ instead of $\<\{g\}\>$.  
For small finite sets, like $\{x, y\}$, we often write, 
$\<x, y\>$ instead of $\<\{x, y\}\>$.

A {\bf one-generated subgroup} is a subgroup generated by one
element, such as $\<g\>$.  A one-generate subgroup is also called a {\bf cyclic
subgroup}.  A {\bf two-generate subgroup} is a
subgroup $\<x, y\>$ that is generated by two elements, $x$ and $y$.
An {\bf $n$-generated subgroup} is a
subgroup of the form $\<x_1, \dots, x_n\>$, generated by the $n$ elements,
$x_1, \dots, x_n$.

Let $G$ be a group and let $H$ be a subgroup of $G$.
It is important to note the distinction between the following two statements:
\begin{enumerate}
\item ``The cyclic subgroup $H$ has two generators $x$ and $y$.''
\item ``The subgroup $H$ is generated by two elements $x$ and $y$.''
\end{enumerate}
The first sentence means $H = \<x\> = \<y\>$.  That is, you can take either $x$
or $y$ as the generator of $H$.  

The second sentence above means something entirely
different, namely, $H = \<x, y\>$.  This says that the smallest subgroup of $G$
that contains both $x$ and $y$ is $H$.  It may or may not be the case that $H$
is cyclic in this case.  The notation $H = \<x, y\>$ simply means that $H$ can
be generated by two elements.  It's possible that we could find
an element that generates $H$ all by itself.  That is, we
may have $H = \<g\> = \<x, y\>$.  

\bigskip

{\bf Examples}
\begin{enumerate}
\item 
Consider the subgroup $H = \{e, (1,2,3), (1,3,2)\}$ of $A_4$,
which can be generated by either one (or both)
of its non-identity elements:
$H = \<(1,2,3)\> = \<(1,3,2)\>$.

\item
Continuing with the last example, we could write $H = \<(1,2,3), (1,3,2)\>$.
Here we have thrown in a redundant generator, which is harmless, but not helpful
because it doesn't call attention to an important feature of$H$---namely, that it is
one-generated, i.e., cyclic.

\item As mentioned above, the subgroup of $\Z_{12}$ generated by the set $\{4, 6\}$ is 
$\{0, 2, 4, 6, 8, 10\}$.
To see this, note that, if 4 and 6 belong to a
subgroup of $\Z_{12}$, then so must $4+4=8$ and $4+6=10$ and $6+6=0$ and
$4+4+6=2$.

\item 
Suppose $G = \<a\>$ is a cyclic group, suppose $x = a^6$ and $y = a^8$.
Then 
\[
H = \<x, y\> = \<a^6, a^8\> = \<a^2\>.
\]

\end{enumerate}
See also the CyclicGroupSupplement.pdf document and CyclicGroupExercises.pdf,
especially Exercise 6.

\end{document}

