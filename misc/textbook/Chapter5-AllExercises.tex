% Example LaTeX document for GP111 - note % sign indicates a comment
\documentclass[12pt,reqno]{amsart}
\usepackage[top=1.5cm, left=1.5cm,right=1.5cm,bottom=1.5cm]{geometry}
\renewcommand{\baselinestretch}{1.2}
\usepackage{amsmath}
\usepackage{amssymb}
\usepackage{color,hyperref,enumerate,multicol}
\definecolor{darkblue}{rgb}{0.0,0.0,0.3}
\hypersetup{colorlinks,breaklinks,
            linkcolor=darkblue,urlcolor=darkblue,
            anchorcolor=darkblue,citecolor=darkblue}
            
\usepackage{algorithm}
\usepackage{algorithmic}
\pagestyle{empty}
\newcommand{\N}{\ensuremath{\mathbb{N}}}
\newcommand{\Z}{\ensuremath{\mathbb{Z}}}
\newcommand{\R}{\ensuremath{\mathbb{R}}}
\newcommand{\meet}{\ensuremath{\wedge}}
\newcommand{\Meet}{\ensuremath{\bigwedge}}
\newcommand{\join}{\ensuremath{\vee}}
\renewcommand{\emptyset}{\ensuremath{\varnothing}}
\renewcommand{\subset}{\ensuremath{\subsetneq}}
\newcommand{\boldemph}{\emph}
\newcommand{\lcm}{\operatorname{lcm}}
\newcommand{\cis}{\ensuremath{\operatorname{cis}}}

\begin{document}
\thispagestyle{empty}

\noindent \textbf{Abstract Algebra} \hskip3cm {\bf Thomas Judson} \hfill {\bf Chapter 5 Exercises}
\medskip

\begin{enumerate}[{\bf 1.}]
\item
 
%*****************Calculations********************
 
 
\item %1
Write the following permutations in cycle notation.
\begin{multicols}{2}
\begin{enumerate}
 
\item
\[
\begin{pmatrix}
1 & 2 & 3 & 4 & 5 \\
2 & 4 & 1 & 5 & 3
\end{pmatrix}
\]

\item
\[
\begin{pmatrix}
1 & 2 & 3 & 4 & 5 \\
4 & 2 & 5 & 1 & 3
\end{pmatrix}
\]

\item
\[
\begin{pmatrix}
1 & 2 & 3 & 4 & 5 \\
3 & 5 & 1 & 4 & 2
\end{pmatrix}
\]

\item
\[
\begin{pmatrix}
1 & 2 & 3 & 4 & 5 \\
1 & 4 & 3 & 2 & 5
\end{pmatrix}
\]

\end{enumerate}
\end{multicols}


 
 \item  %2
Compute each of the following.
\begin{multicols}{2}
\begin{enumerate}
 
\item
$(1345)(234)$
  
\item
$(12)(1253)$

\item
$(143)(23)(24)$

\item
$(1423)(34)(56)(1324)$

\item
$(1254)(13)(25)$

 
\item
$(1254) (13)(25)^2$
 
\item
$(1254)^{-1} (123)(45) (1254)$
  
\item
$(1254)^2 (123)(45)$
 
\item
$(123)(45) (1254)^{-2}$

\item
$(1254)^{100}$
 
\item
$|(1254)|$

  
\item
$|(1254)^2|$
  
\item
$(12)^{-1}$

\item
$(12537)^{-1}$
 
\item
$[(12)(34)(12)(47)]^{-1}$

\item
$[(1235)(467)]^{-1}$
 
\end{enumerate}
\end{multicols}
 
 
 \item %3
Express the following permutations as products of transpositions and
identify them as even or odd. 
\begin{multicols}{2}
\begin{enumerate}
 
\item
$(14356)$

 \item
$(156)(234)$
 
 \item
$(1426)(142)$
 
 \item
$(17254)(1423)(154632)$
 
 \item
$(142637)$
 
\end{enumerate}
\end{multicols}


 
\item %5
Find $(a_1, a_2, \ldots, a_n)^{-1}$.
 
\item %6
List all of the subgroups of $S_4$. Find each of the following sets. 
\begin{enumerate}
 
 \item
$\{ \sigma \in S_4 : \sigma(1) = 3 \}$
 
 \item
$\{ \sigma \in S_4 : \sigma(2) = 2 \}$
 
 \item
$\{ \sigma \in S_4 : \sigma(1) = 3 \mbox{ and } \sigma(2) =
2 \}$
 
\end{enumerate}
Are any of these sets subgroups of $S_4$?
 
 
\item
Find all of the subgroups in $A_4$. What is the order of each
subgroup? 
 
 
\item
Find all possible orders of elements in $S_7$ and $A_7$.
 
 
\item
Show that $A_{10}$ contains an element of order 15.
 
 
\item
Does $A_8$ contain an element of order 26?
 
 
\item %7
Find an element of largest order in $S_n$ for $n = 3, \ldots, 10$. 
 
 
\item
What are the possible cycle structures of elements of $A_5$? What
about $A_6$? 
 
 
\item
Let $\sigma \in S_n$ have order $n$. Show that for all integers $i$
and $j$, $\sigma^i = \sigma^j$ if and only if $i \equiv j \pmod{n}$. 
 
 
\item\label{permute:OrderProductCycles}
Let $\sigma = \sigma_1 \cdots \sigma_m \in S_n$ be the product of
disjoint cycles. Prove that the order of $\sigma$ is the least common
multiple of the lengths of the cycles $\sigma_1, \ldots, \sigma_m$.
 
 
\item
Using cycle notation, list the elements in $D_5$.  What are $r$ and
$s$?  Write every element as a product of $r$ and $s$.
 
 
\item
If the diagonals of a cube are labeled as Figure~\ref{motions}, to
which motion of the cube does the permutation $(12)(34)$ correspond?
What about the other permutations of the diagonals?
 
 
\item
Find the group of rigid motions of a tetrahedron.  Show that this is
the same group as $A_4$. 
 
 
%******************************Theory********************
 
 
\item
Prove that $S_n$ is nonabelian for $n \geq 3$.
 
 
\item
Show that $A_n$ is nonabelian for $n \geq 4$.
 
 
\item
Prove that $D_n$ is nonabelian for $n \geq 3$.
 
 
\item
Let $\sigma \in S_n$. Prove that $\sigma$ can be written as the
product of at most $n-1$ transpositions. 
 
 
\item
Let $\sigma \in S_n$. If $\sigma$ is not a cycle, prove that $\sigma$
can be written as the product of at most $n-2$ transpositions.
 
 
\item
If $\sigma$ can be expressed as an odd number of transpositions, show
that any other product of transpositions equaling $\sigma$ must also
be odd. 
 
 
\item
If $\sigma$ is a cycle of odd length, prove that $\sigma^2$ is also a
cycle. 
 
 
\item
Show that a 3-cycle is an even permutation.
 
 
\item
Prove that in $A_n$ with $n \geq 3$, any permutation is a product of
cycles of length~3.  
 
 
\item
Prove that any element in $S_n$ can be written as a finite product of
the following permutations.
\begin{enumerate}
 
 \item
$(1 2), (13), \ldots, (1n)$
 
 \item
$(1 2), (23), \ldots, (n- 1,n)$
 
 \item
$(12), (1 2 \ldots n )$
 
\end{enumerate}
 
 
\item
Let $G$ be a group and define a map $\lambda_g : G \rightarrow G$ by
$\lambda_g(a) = g a$.  Prove that $\lambda_g$ is a permutation of $G$.
 
 
 
\item
Prove that there exist $n!$ permutations of a set containing $n$
elements. 
 
 
\item
Recall that the \boldemph{center}\index{Group!center of} of a group $G$ is
\[
Z(G) = \{ g \in G : \mbox{$gx = xg$ for all $x \in G$} \}.
\]
Find the center of $D_8$. What about the center of $D_{10}$? What is
the center of $D_n$? 
 
 
\item
Let $\tau = (a_1, a_2, \ldots, a_k)$ be a cycle of length $k$.
\begin{enumerate}
 
 \item
Prove that if $\sigma$ is any permutation, then
\[
\sigma \tau \sigma^{-1 } = ( \sigma(a_1), \sigma(a_2), \ldots,
\sigma(a_k))
\]
is a cycle of length $k$.
 
 \item
Let $\mu$ be a cycle of length $k$. Prove that there is a permutation
$\sigma$ such that $\sigma \tau \sigma^{-1 } = \mu$.
 
\end{enumerate}
 
 
\item
For $\alpha$ and $\beta$ in $S_n$, define $\alpha \sim \beta$ if there
exists an $\sigma \in S_n$ such that $\sigma \alpha \sigma^{-1} =
\beta$.  Show that $\sim$ is an equivalence relation on $S_n$.
 
 
\item
Let $\sigma \in S_X$. If $\sigma^n(x) = y$, we will say that $x \sim
y$. 
\begin{enumerate}
 
 \item
Show that $\sim$ is an equivalence relation on $X$.
 
 \item
If $\sigma \in A_n$ and $\tau \in S_n$, show that $\tau^{-1} \sigma
\tau \in A_n$. 
 
\item
Define the \boldemph{orbit}\index{Orbit} of $x \in X$ under $\sigma \in
S_X$ to be the set 
\[
{\mathcal O}_{x, \sigma} = \{ y : x \sim y  \}.
\]
Compute the orbits of $\alpha, \beta, \gamma$ where
\begin{align*}
\alpha & = (1254) \\
\beta & = (123)(45)\\
\gamma & = (13)(25).
\end{align*}
 
 \item
If ${\mathcal O}_{x, \sigma} \cap {\mathcal O}_{y, \sigma} \neq \emptyset$,
prove that ${\mathcal O}_{x, \sigma} = {\mathcal O}_{y, \sigma}$.  The orbits
under a permutation $\sigma$ are the equivalence classes corresponding
to the equivalence relation $\sim$.
 
 
\item
A subgroup $H$ of $S_X$ is \boldemph{
transitive}\index{Subgroup!transitive} if for every $x, y \in X$, 
there exists a $\sigma \in H$ such that $\sigma(x) =y$. Prove that
$\langle \sigma \rangle$ is transitive if and only if ${\mathcal O}_{x,
\sigma} = X$ for some $x \in X$. 
 
 
\end{enumerate}
 
 
\item
Let $\alpha \in S_n$ for $n \geq 3$. If $\alpha \beta = \beta \alpha$
for all $\beta \in S_n$, prove that $\alpha$ must be the identity
permutation; hence, the center of $S_n$ is the trivial subgroup. 
 
 
\item
If $\alpha$ is even, prove that $\alpha^{-1}$ is also even. Does a
corresponding result hold if $\alpha$ is odd? 
 
 
\item
Show that  $\alpha^{-1} \beta^{-1} \alpha \beta$ is even for $\alpha,
\beta \in S_n$. 
 
 
\item
Let $r$ and $s$ be the elements in $D_n$ described in Theorem~\ref{permute:Dn_generator_theorem}.
\begin{enumerate}
 
 \item
Show that $srs = r^{-1}$.
 
 \item
Show that $r^k s = s r^{-k}$ in $D_n$.
 
 \item
Prove that the order of $r^k \in D_n$ is $n / \gcd(k,n)$.
  
\end{enumerate}
 
 
\end{enumerate}

\end{document}
