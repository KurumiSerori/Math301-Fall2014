Let $X$ be a finite set and consider the set $X^X$ of all maps from $X$ to
itself, which, when endowed with composition of maps and the identity mapping,
forms a monoid, $\<X^X; \circ, \id{X}\>$.  The submonoid $S_X$ of all bijective
maps in $X^X$ is a group, the \emph{symmetric group on $X$}.  When the
underlying set isn't important, we write $S_n$ to denote the generic
symmetric group on an $n$-element set. 

If we have defined some set $F$ of basic operations on $X$, so that
$\bX = \<X; F\>$ is an algebra, then two other important submonoids of
$X^X$ are $\End(\bX)$, the set of maps in $X^X$ which respect all 
operations in $F$, and $\Aut(\bX)$, the set of bijective maps in  $X^X$ which
respect all operations in $F$.  It is apparent from the definition that
 $\Aut(\bX)= S_X \cap \End(\bX)$, and  $\Aut(\bX)$ is a submonoid of $\End(\bX)$
 and a subgroup of $S_X$.  These four fundamental monoids
 associated with the algebra $\bX$ are shown in the diagram below. 

\begin{center}
  \begin{tikzpicture}[scale=.7]
%    \node (Aut) at (0,0) [fill,circle,inner sep=1pt] {};
    \draw[font=\small] (0,0) node {$\Aut(\bX)$};
%    \node (End) at (-2,2) [fill,circle,inner sep=1pt] {};
    \draw[font=\small] (-2,2) node {$\End(\bX)$};
%    \node (Sx) at (2,2) [fill,circle,inner sep=1pt] {};
%    \draw[font=\small] (2,2) node {$S_X$};
    \draw (2,2) node {$S_X$};
%    \node (XX) at (0,4) [fill,circle,inner sep=1pt] {};
%    \draw[font=\small] (0,4) node {$X^X$};
    \draw (0,4) node {$X^X$};
    \draw[font=\small] (-1,1) node {\rotatebox[origin=c]{130}{$\leq$}};
    \draw[font=\small] (1,1) node {\rotatebox[origin=c]{45}{$\leq$}};
    \draw[font=\small] (1,3) node {\rotatebox[origin=c]{130}{$\leq$}};
    \draw[font=\small] (-1,3) node {\rotatebox[origin=c]{45}{$\leq$}};

%    \draw[semithick]    (Aut) to (End) to (XX) to (Sx) to (Aut);
  \end{tikzpicture}
\end{center}


Given a finite group $G$, and an algebra $\bX = \<X; F\>$, a
\emph{representation} of $G$ on $\bX$ is a group homomorphism
from $G$ into $\Aut(\bX)$.  That is, a representation of $G$ is a mapping
$\varphi : G \rightarrow \Aut(\bX)$ which satisfies $\varphi(g_1 g_2) =
\varphi(g_1) \circ \varphi(g_2)$, where (as above) $\circ$ denotes composition
of maps in $\Aut(\bX)$.

Thus, a representation defines an action by $G$ on the set $X$: $\bar{g} x =
\varphi(g)(x)$.  If $\bar{G} = \varphi[G]$ denotes the image of $G$ under
$\varphi$, then $\< X; \bar{G}\>$ is a G-set.\footnote{More generally, a G-set is
  sometimes defined to be a pair $(X, \varphi)$, where $\varphi$ is a homomorphism from
  a group into the symmetric group $S_X$; see e.g.~\cite{Suzuki:1982}.}  
The action is called
\emph{transitive} iff for each pair $x, y \in X$ there is some $g\in G$ such
that $\varphi(g)(x) = y$. The representation $\varphi$ is called \emph{faithful}
iff it is a monomorphism, in which case $G$ is isomorphic to its image under
$\varphi$, which is a subgroup of $\Aut(\bX)$.  We also say, in this case, that
the group acts faithfully, and call it a \emph{permutation group}.
A group which acts transitively on some set is called a \emph{transitive group}.
Without specifying the set, however, this term is meaningless, since
every group acts transitively on some sets and intransitively on others.  
A representation $\varphi$ is called transitive iff the resulting action is transitive.

Two special cases are almost always what one means when one speaks of a
representation of a finite group.  They are the so called
\begin{itemize}
\item \emph{linear representations}, where $\bX = \<X; +, \circ, -, 0, 1, \F\>$ is a finite dimensional vector
  space over a field $\F$, so $\Aut(\bX)$ is the set of invertible matrices with entries from $\F$;
\item \emph{permutation representations}, where $\bX = X$ is just a set, so $\Aut(\bX) = S_X$.
\end{itemize}

% Given a group $G$, there is a set of natural permutation representations of $G$
% associated with the (conjugacy classes of) subgroups of $G$.  Let $H$ be any
% subgroup of $G$ and consider the set $X = G/H = \{H, x_1H, \dots, x_{r-1}H\}$ of
% left cosets of $H$. 
The following elementary theorem tells us precisely when a particular group $G$
has a transitive permutation representation on a set of size $n$.
The theorem is easy to prove.\footnote{See, e.g., \cite{Suzuki:1982} Theorem
  7.16.}
\begin{theorem}
  Let $G$ be a group.  The following three conditions are equivalent.
  \begin{enumerate}[(i)]
  \item There is a transitive permutation representation of $G$ on a set of size
    $n$.
  \item There is a homomorphism from $G$ into $S_n$ such that the image of $G$
    is transitive. 
  \item The group $G$ has a subgroup of index $n$.
  \end{enumerate}
\end{theorem}

\newcommand{\Core}{\ensuremath{\mathrm{Core}}}
For a given group $G$, and any subgroup $H< G$,
we define a transitive permutation representation of $G$, which we
denote $\rho_H$.  Specifically, $\rho_H$ is a group homomorphism from $G$ into
the symmetric group $\Sym(G/H)$ of permutations on the set $G/H = \{H, Hx_1,
Hx_2, \dots \}$ of \emph{right} cosets of $H$ in $G$.
The action is simply right-multiplication by elements of $G$. That is:
% \footnote{We could have defined the action, $\lambda_H: G
%   \rightarrow G/H$, on the \emph{left} cosets of $H$ in $G$, where $\lambda_H(g)$
%   is \emph{left}-multiplication by $g$.  
\[
\rho_H : G \rightarrow \Sym(G/H), \quad \text{ where } \quad 
\rho_H(g)(Hx)= Hxg.
\]

With this set-up, to check the homomorphism property of $\rho_H$, 
we should write the permutation mappings in $\Sym(G/H)$ on the
right of their arguments, as in $Hx \rho_H(g) = Hxg$.  For then we have
% $Hx \rho_H(g_1 g_2) = Hx (g_1 g_2) = Hx g_1 g_2 = Hx\rho_H(g_1)\rho_H(g_2)$; 
% i.e.~$\rho_H(g_1 g_2) = \rho_H(g_1)\rho_H(g_2)$.}
\[
Hx \rho_H(g_1 g_2) = Hx (g_1 g_2) = Hx g_1 g_2 = Hx\rho_H(g_1)\rho_H(g_2);
\] 
i.e.~$\rho_H(g_1 g_2) = \rho_H(g_1)\rho_H(g_2)$.

For each $Hx \in G/H$, the \emph{point stabilizer} of $Hx$ is 
\[
G_{Hx} = \{g\in G : Hxg = Hx \} = 
\{g\in G : Hxgx^{-1}  = H \} = 
%\{x g x^{-1}\in G : g H = H \} = 
x^{-1} G_H x  = x^{-1} H x = H^x,
\]
so the kernel of the homomorphism $\rho_H$ is 
\[
\ker \rho_H = \{g\in G : \forall x \in G,\; Hxg = Hx \} = 
%\bigcap_{x\in G} \{g\in G : x^{-1}gx H = H \} = 
\bigcap_{x\in G} x^{-1} H x = \bigcap_{x\in G} H^x.
\]
Note that $\ker \rho_H$ is the largest normal subgroup of $G$ 
contained in $H$, also known as the \emph{core of $H$ in $G$}, which we denote
by 
\[
\Core_G(H) = \bigcap_{x\in G} H^x.
\]

Next we describe (up to equivalence) all transitive permutation
representations of a given group $G$.  
We call two representations (or actions) \emph{equivalent}
iff the associated $G$-sets are isomorphic. 
The foregoing implies that every transitive permutation representation of $G$ is
equivalent to $\rho_H$ for some subgroup $H < G$.  The following
lemma\footnote{Lemma 1.6B of \cite{Dixon:1996}.} 
shows that we need only consider a single representative $H$ from each of the
conjugacy classes of subgroups.  

\begin{lemma}
Suppose $G$ acts transitively on two sets,
$A$ and $B$.  Fix $a\in A$ and let $G_a$ be the stabilizer of $a$ (under the first
action).  Then the two actions are equivalent
if and only if the subgroup $G_a$ is also a stabilizer under the second action
of some point $b\in B$. 
\end{lemma}

The point stabilizers of the action $\rho_H$ described above are the
conjugates of $H$ in $G$.  Therefore, the lemma implies that, for any two
subgroups $H, K \leq G$, the representations $\rho_H$ and $\rho_K$ are
equivalent precisely when $K = x^{-1} Hx$ for some $x\in G$. 
Hence, the transitive permutation representations of $G$ are given, up to
equivalence, by $\rho_{K_i}$ as $K_i$ runs over a set of representatives of
conjugacy classes of subgroups of $G$.   

