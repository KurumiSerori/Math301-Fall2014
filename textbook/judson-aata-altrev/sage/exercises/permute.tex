%%%%(c)
%%%%(c)  This file is a portion of the source for the textbook
%%%%(c)
%%%%(c)    Abstract Algebra: Theory and Applications
%%%%(c)    by Thomas W. Judson
%%%%(c)
%%%%(c)    Sage Material
%%%%(c)    Copyright 2011 by Robert A. Beezer
%%%%(c)
%%%%(c)  See the file COPYING.txt for copying conditions
%%%%(c)
%%%%(c)
These exercises are designed to help you become familiar with permutation groups in Sage.
%
\begin{sageverbatim}\end{sageverbatim}
%
\sageexercise{1}%
Create the full symmetric group $S_{10}$ with the command \verb?G = SymmetricGroup(10)?.
\begin{sageverbatim}\end{sageverbatim}
%
\sageexercise{2}
Create elements of \verb?G? with the following (varying) syntax.  Pay attention to commas, quotes, brackets, parentheses.  The first two use a string (characters) as input, mimicking the way we write permuations (but with commas).  The second two use a list of tuples.\par\noindent
\verb?a = G("(5,7,2,9,3,1,8)")?\\
\verb?b = G("(1,3)(4,5)")?\\
\verb?c = G([(1,2),(3,4)])?\\
\verb?d = G([(1,3),(2,5,8),(4,6,7,9,10)])?\par\noindent
%
(a)  Compute $a^3$, $bc$, $ad^{-1}b$.\par\noindent
%
(b)  Compute the orders of each of these four individual elements (\verb?a? through \verb?d?) using a single permutation group element method.\par\noindent
%
(c)  Use the permutation group element method \verb?.sign()? to determine if $a,b,c,d$ are even or odd permutations.\par\noindent
%
(d)  Create two cyclic subgroups of $G$ with the commands:
%
\begin{itemize}
\item\verb?H = G.subgroup([a])?
\item\verb?K = G.subgroup([d])?
\end{itemize}
%
List, and study, the elements of each subgroup.  Without using Sage, list the order of each subgroup of $K$.  Then use Sage to construct a subgroup of $K$ with order 10.\par\noindent
%
(e)  More complicated subgroups can be formed by using two or more generators.  Construct a subgroup $L$ of $G$ with the command \verb?L = G.subgroup([b,c])?.  Compute the order of $L$ and list all of the elements of $L$.
\begin{sageverbatim}\end{sageverbatim}
%
\sageexercise{3}
Construct the group of symmetries of the tetrahedron (also the alternating group on 4 symbols, $A_4$) with the command \verb?A=AlternatingGroup(4)?.  Using tools such as orders of elements, and generators of subgroups, see if you can find \emph{all of} the subgroups of $A_4$ (each one exactly once).  Do this without using the \verb?.subgroups()? method to justify the correctness of your answer (though it might be a convenient way to check your work).\par
%
Provide a nice summary as your answer - not just piles of output.  So use Sage as a tool, as needed, but basically your answer will be a concise paragraph and/or table.  This is the one part of this assignment without clear, precise directions, so spend some time on this portion to get it right.  Hint: no subgroup of $A_4$ requires more than two generators.
\begin{sageverbatim}\end{sageverbatim}
%
\sageexercise{4}
Save your work, and then see if you can crash your Sage session with the commands.  Do not submit the list of elements of \verb?N? as part of your submitted worksheet.
%
\begin{itemize}
\item\verb?N = G.subgroup([b,d])?
\item\verb?N.list()?
\end{itemize}
%
How big is $N$?
\begin{sageverbatim}\end{sageverbatim}
%
\sageexercise{5}
Answer the five questions above about the permutations of the cube expressed as permutations of the 8 vertices.
\begin{sageverbatim}\end{sageverbatim}
%
